\newpage
\section{Diskussion}
Die Verhältnisse der Frequenzen liegen nah an den theoretisch berechnteten Werten,ihre Fehler liegen zwischen 4\% und 45\% wobei der Fehler von 45\% eine Ausnahme ist und auf schlechtes ablesen der Maxima zurücklgeführt werden kann.

Auch der Velauf der Frequenzen in Abhängigkeit von den $C_k$ zeigt nur geringe Abweichungen von den theoretischen Werten.

Bei der Abhängigkeit des Stromes $I_2$ von den Frequenzen fällt auf dass die Maxima des Experiments weit unter den theoretischen Maxima liegen.
Die Abstände der theoretischen Maxima stimmen zwar ungefähr mit denen der Messung überein, aber auch die Lage der Werte ist leicht verschoben,
diese Fehler sind durch einen Fehler beim Ablesen oder Fehler in der Umrechnung zu erklären.
Auch bei der Betrachtung der Abweichungen $a_I$ fällt auf, dass diese sehr groß sind und somit nicht für ein Aussagekräftiges Versuchsergebnis sprechen.

Die Fundamentalfrequenzen liegen wie im Plot \ref{fig:frequenzverlauf} erkennbar nah an den Theoriewerten, auch die Abweichungen die zwischen 1\% und 8\% liegen verifizieren unsere Werte als plausibel.

\label{sec:Diskussion}
