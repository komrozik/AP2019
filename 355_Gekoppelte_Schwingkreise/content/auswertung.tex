\newpage
\section{Auswertung}
%Anmerkung von Marcel aus Errinnerung
% 4. Vorbereitende Justierung - die ermittelte Resonanzfrquenz soll mit Theoretischen veglichen werden
%                               und mögliche Abweichungen diskutiert werden
\subsection{Berechnung Frequenzen}
Zu Beginn werden die Frequenzen der Fundamentalschwingungen theoretisch bestimmt und ihr Verhältnis wird mit den experimentell bestimmten Werten verglichen.
Mit \ref{eqn:Fundamental_Frequenzen1} und \ref{eqn:Fundamental_Frequenzen2} ergibt sich:

\begin{table}[H]
    \centering
    \begin{tabular}{c c c}
        \toprule
        $C_k \; / \; \si{\nano \farad}$    &   $\nu^- \; / \; \si{\kilo \hertz}$  &  $\nu^+ \; / \; \si{\kilo \hertz}$ \\
        \midrule
        1,01 & \num{47,5\pm 0,3 }         & \num{30,5 }  \\
        2,03 & \num{40,18\pm 0,24 }     & \num{30,5 }  \\
        3,00 & \num{37,41\pm 0,18 }     & \num{30,5 }  \\
        4,00 & \num{35,85\pm 0,14 }     & \num{30,5 }  \\
        5,02 & \num{34,854\pm 0,119 }   & \num{30,5 }  \\
        6,47 & \num{33,94\pm 0,09 }     & \num{30,5 }  \\
        8,00 & \num{33,33\pm 0,07 }     & \num{30,5 }  \\
        9,99 & \num{32,80\pm 0,06 }     & \num{30,5 }  \\
        \bottomrule
    \end{tabular}
    \caption{Frequenzen der Fundamentalschwingungen}
    \label{tab:Frequenzen}
\end{table}


\begin{table}[H]
    \centering
    \begin{tabular}{c c c c}
        \toprule
        $C_k$    &   Verhältnis der Frequenzen & Experiment & Abweichung von Theorie und Experiment\\
        \midrule
        1,01 & \num{2,30\pm 0,04}     & \num{2}        & \num{0,131\pm 0,015}\\        
        2,03 & \num{3,67\pm 0,08}     & \num{2}        & \num{0,4556\pm 0,0120}\\
        3,00 & \num{4,957\pm 0,118}   & \num{4}        & \num{0,193\pm 0,019}\\
        4,00 & \num{6,27\pm 0,15}     & \num{6}        & \num{0,043\pm 0,024}\\
        5,02 & \num{7,61\pm 0,19}     & \num{7}        & \num{0,080\pm 0,023}\\
        6,47 & \num{9,51\pm 0,25}     & \num{9}        & \num{0,053\pm 0,025}\\
        8,00 & \num{11,5\pm 0,3}      & \num{11}       & \num{0,044\pm 0,026}\\       
        9,99 & \num{14,1\pm 0,3}      & \num{13}       & \num{0,078\pm 0,025}\\
       \bottomrule
    \end{tabular}
    \caption{Frequenzverhältnisse der Fundametalschwingungen und Abweichung der Messwerte von der Theorie}
    \label{tab:Frequenzverhältnisse}
\end{table}

\begin{figure}[H]
    \centering
    \includegraphics[width = \textwidth]{build/Frequenzverlauf.pdf}
    \caption{Verlauf der Frequenzen}
    \label{fig:frequenzverlauf}
\end{figure}

\subsection{Stromabhängigkeit von der Frequenz}
Im folgenden wird die Abhängigkeit des Stromes von den Frequenzen für verschiedene $C_k$ dargestellt.
\begin{figure}
    \centering
    \includegraphics[width = \textwidth]{build/Stromverlauf.pdf}
    \caption{Stromverlauf in Abhängigkeit der Frequenzen}
    \label{fig:Stromverlauf}
\end{figure}



\label{sec:Auswertung}