\newpage
\section{Auswertung}
%Anmerkung von Marcel aus Errinnerung
% 4. Vorbereitende Justierung - die ermittelte Resonanzfrquenz soll mit Theoretischen veglichen werden
%                               und mögliche Abweichungen diskutiert werden

\subsection{Berechnung Frequenzen}Für die Auswertung der Messdaten werden zuerst die theoretischen Werte der Fundamentalschwingungen bestimmt.
Nach \ref{eqn:Fundamental_Frequenzen1} und \ref{eqn:Fundamental_Frequenzen2} gilt:
\begin{align*}
    \nu_+ &= \frac{1}{2\pi} \frac{1}{\sqrt{L\cdot \left(C + C_{Sp}\right)}}\\
    \nu_- &= \frac{1}{2\pi} \frac{1}{\sqrt{L\cdot \left(\left(\frac{1}{C}+\frac{1}{C_k}\right)^{-1} + C_{Sp}\right)}}\\
\end{align*}
Das Verhältnis der beiden Frequenzen bestimmt sich dann mit \ref{eqn:Verhältnis}.
Mithilfe der Messwerte wird nun die Abweichung der Messwerte von den Theoriewerten bestimmt.
\begin{equation}
    a_n = \frac{|n_{theorie} -n_{experiment}|}{n_{theorie}}
    \label{eqn:Abweichung}
\end{equation}
Die Abweichung der verschiedenen Fundamentalfrequenzen lässt sich dementsprechend mit $\nu_{theoretisch}$ und $\nu_{experiment}$ berechnen.

Zu Beginn werden die Frequenzen der Fundamentalschwingungen $\nu_+$ und $\nu_-$ theoretisch bestimmt und die Abweichungen $a_{\nu_+}$ und $a_{\nu_-}$ der experimentellen Werte bestimmt.
\begin{table}[H]
    \centering
    \begin{tabular}{c c c c c c c}
        \toprule
        $C_k \; / \; \si{\nano \farad}$    &   $\nu^- \; / \; \si{\kilo \hertz}$  &  $\nu^+ \; / \; \si{\kilo \hertz}$ & $\nu_{-,theorie} \; / \; \si{\kilo \hertz} $ & $\nu_{+,theorie} \; / \; \si{\kilo \hertz}$ & $a_{\nu_+}$ & $a_{\nu_-}$\\
        \midrule
        1,01  &  \num{47,2} & \num{30,5} & \num{47,5\pm 0,3 }       & \num{30,557 }    &  \num{0,0018}    & \num{0,006 \pm 0,008}\\
        2,03  &  \num{40,3} & \num{30,5} & \num{40,18\pm 0,24 }     & \num{30,557 }    &  \num{0,0018}    & \num{0,002 \pm 0,006}\\
        3,00  &  \num{37,2} & \num{30,5} & \num{37,41\pm 0,18 }     & \num{30,557 }    &  \num{0,0018}    & \num{0,005 \pm 0,004}\\
        4,00  &  \num{35,9} & \num{30,5} & \num{35,85\pm 0,14 }     & \num{30,557 }    &  \num{0,0018}    & \num{0,001 \pm 0,004}\\
        5,02  &  \num{34,9} & \num{30,5} & \num{34,854\pm 0,119 }   & \num{30,557 }    &  \num{0,0018}    & \num{0,001 \pm 0,003}\\
        6,47  &  \num{34,0} & \num{30,5} & \num{33,94\pm 0,09 }     & \num{30,557 }    &  \num{0,0018}    & \num{0,001 \pm 0,002}\\
        8,00  &  \num{33,4} & \num{30,5} & \num{33,33\pm 0,07 }     & \num{30,557 }    &  \num{0,0018}    & \num{0,002 \pm 0,002}\\
        9,99  &  \num{32,9} & \num{30,5} & \num{32,80\pm 0,06 }     & \num{30,557 }    &  \num{0,0018}    & \num{0,002 \pm 0,001}\\
        \bottomrule
    \end{tabular}
    \caption{Frequenzen der Fundamentalschwingungen}
    \label{tab:Frequenzen}
\end{table}

Nun werden noch die Verhältnisse der Frequenzen nach \ref{eqn:Verhältnis} bestimmt und ihre Abweichung nach \ref{eqn:Abweichung}.
\begin{table}[H]
    \centering
    \begin{tabular}{c c c c}
        \toprule
        $C_k$    &   $n_{theoretisch}$ & $n_{experiment}$ & $a_n$\\
        \midrule
        1,01 & \num{2,30\pm 0,04}     & \num{2}        & \num{0,131\pm 0,015}\\        
        2,03 & \num{3,67\pm 0,08}     & \num{2}        & \num{0,4556\pm 0,0120}\\
        3,00 & \num{4,957\pm 0,118}   & \num{4}        & \num{0,193\pm 0,019}\\
        4,00 & \num{6,27\pm 0,15}     & \num{6}        & \num{0,043\pm 0,024}\\
        5,02 & \num{7,61\pm 0,19}     & \num{7}        & \num{0,080\pm 0,023}\\
        6,47 & \num{9,51\pm 0,25}     & \num{9}        & \num{0,053\pm 0,025}\\
        8,00 & \num{11,5\pm 0,3}      & \num{11}       & \num{0,044\pm 0,026}\\       
        9,99 & \num{14,1\pm 0,3}      & \num{13}       & \num{0,078\pm 0,025}\\
       \bottomrule
    \end{tabular}
    \caption{Frequenzverhältnisse der Fundametalschwingungen und Abweichung der Messwerte von der Theorie}
    \label{tab:Frequenzverhältnisse}
\end{table}

\begin{figure}[H]
    \centering
    \includegraphics[width = \textwidth]{build/Frequenzverlauf.pdf}
    \caption{Verlauf der Frequenzen}
    \label{fig:frequenzverlauf}
\end{figure}

\subsection{Stromabhängigkeit von der Frequenz}
Im folgenden wird die Abhängigkeit des Stromes von den Frequenzen für verschiedene $C_k$ dargestellt.
Hierbei lässt sich der Strom $I$ durch \ref{eqn:StromI} darstellen wie in \ref{fig:Stromverlauf} zu sehen.
\begin{figure}
    \centering
    \includegraphics[width = 0.75\textwidth]{build/Stromverlauf.pdf}
    \caption{Stromverlauf in Abhängigkeit der Frequenzen}
    \label{fig:Stromverlauf}
\end{figure}

\begin{table}[H]
    \centering
    \begin{tabular}{c c c c}
        \toprule
        $C_k$    &   $I_{theoretisch}$ & $I_{experiment}$ & $a_I$\\
        \midrule %
        1,01 & \num{0.0057}         & \num{0.0083}        & \num{0.4464}     \\        
        2,03 & \num{0.0113}         & \num{0.0083}        & \num{0.2677}     \\
        3,00 & \num{0.0122}         & \num{0.0083}        & \num{0.3171}     \\
        4,00 & \num{0.0166}         & \num{0.0083}        & \num{0.4983}     \\
        5,02 & \num{0.0186}         & \num{0.0083}        & \num{0.5538}     \\
        6,47 & \num{0.0213}         & \num{0.0083}        & \num{0.6097}     \\
        8,00 & \num{0.0241}         & \num{0.0083}        & \num{0.6544}     \\       
        9,99 & \num{0.0277}         & \num{0.0083}        & \num{0.6998}     \\
       \bottomrule
    \end{tabular}
    \caption{Frequenzverhältnisse der Fundametalschwingungen und Abweichung der Messwerte von der Theorie}
    \label{tab:Frequenzverhältnisse}
\end{table}


\label{sec:Auswertung}