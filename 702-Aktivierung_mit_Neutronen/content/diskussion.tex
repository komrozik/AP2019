\newpage
\section{Diskussion}
Im Experiment wurde die Nullrate als $(139\pm4)$ Sekunden bestimmt.
Die Genauigkeit der Nullrate kann nicht gut bewertet werden, da hierfür keine Literaturwerte existieren können.
Allerdings wurde im Experiment nur über 300 Sekunden gemessen.
Um den statistischen Fehler der Rate gering zu halten wäre es besser wenn über mindestens 500 Sekunden gemessen werden würde. 
Wie in Tabelle \ref{fig:Tabelle} aufgetragen werden alle Halbwertszeiten mit einer relativen Abweichung von weniger als 20\% bestimmt.
Überraschend ist es, dass die Halbwertszeit für Vanadium auf dem kurzen Intervall viel ungenauer bestimmt wird als auf dem längeren Intervall mit den vermeintlich ungenauen Werten.
Diese Abweichung kann dadurch erklärt werden, dass für das intervall zu wenige Messdaten aufgenommen wurden, und daher statistische Fehler die Messung stark beeinflussen.

\begin{table}
    \centering
    \begin{tabular}{c | c c c}
        \toprule
        & \multicolumn{1}{c}{Vanadium} & \multicolumn{2}{c}{Rhodium}\\
        \cmidrule(lr){2-2}\cmidrule(lr){3-4}
        & $T\;/\;$s & $T_{^{104}\text{Rh}}\;/\;$s & $T_{^{104i}\text{Rh}}\;/\;$s\\
        \midrule
        Literatur\cite{V},\cite{Rh}&224,5\pm0,3&42,3\pm0,4&260,4\pm1,8\\
        Ermittelt&219\pm11&36,2\pm1,7&230\pm50\\
        &186\pm11&&\\
        \midrule
        Abweichung$\;/\;$\%&2,6&13,02&14,49\\
        &17,0&&\\
        \bottomrule
    \end{tabular}
    \caption{Darstellung der Halbwertszeiten $T$ für Vanadium und Rhodium im Vergleich
    mit den Literaturwerten.}
    \label{fig:Tabelle}
\end{table}