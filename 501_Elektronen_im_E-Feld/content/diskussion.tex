\section{Diskussion}
Es zeigt sich aus Tab. \ref{tab:tab1}, dass die Empfindlichkeit $E$ mit zunehmender
Beschleunidungspannung $U_B$ abnimmt.\\

Die Abweichung in $a$ aus den Messwerten zu dem berechneten Wert aus den Konstruktionsparametern
$p, L$ und $d$ ist mit 48.61\% groß.
Ein Teil des Fehler wird darüber einfließen, dass der y-Achsenabschnitt aus \ref{eqn:y} nicht betrachtet wurde.
Allerdings erklärt dies nicht, die größenordnung der Abweichung.
Vermutet wird an systematischer Fehler, der über eine unbekannte Fakorisierung an der Messapperatur
entstanden ist.\\

Die Frequenz der Sinusspannung (vgl. Gl. \ref{eqn:sin}) liegt im Frequenzbereich 
des Sinusgenerators, der eine Frequenz von 80Hz bis 90Hz erzeugt. Auch hier werden stochastische Messunsicherheiten dennoch
eine Rolle gespielt haben, was aufgrund des Ablesen erfolgt sein wird.
\label{sec:Diskussion}
