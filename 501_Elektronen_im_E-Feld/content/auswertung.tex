\section{Auswertung}
\label{sec:Auswertung}
Zunächst werden die Messwerte der Auslenkung gegen die Ablenkspannung aufgetragen und mithilfe einer linearen ausgleichsrechnung approximiert
% PASSEND EINFÜGEN UND AUSKOMMENTIEREN
% \begin{figure}
%     \centering
%     \includegraphics[width = 0.7\textwidth]{plots/??????????.pdf}
%     \caption{Die Messwerte der Auslenkung D werden gegen die Beschleunigungsspannung U aufgetragen.
%              Mit einer linearen Ausgleichsrechnung wird der Verlauf der Werte approximiert.}
% \end{figure}
Mit dem linearen Ausgleich ergibt sich:
\begin{align}
    D(U_d) = m\cdot U_d + b  \nonumber \\
    m = 0 && b = 0 \\
    E_{Empfindlichkeit} = \frac{D}{U_d} = 
\end{align}
Dieser Vorgang wird für die weiteren 4 Beschleunigungsspannungen mit den entsprechenden $U_d$ wiederholt.

....%HIER EINFÜGEN

Nun können  die ermittelten Empfindlichkeiten E gegen den Kehrwert der Beschleunigungsspannungen aufgetragen werden.
% PASSEND EINFÜGEN UND AUSKOMMENTIEREN
% \begin{figure}
%     \centering
%     \includegraphics[width = 0.7\textwidth]{plots/??????????.pdf}
%     \caption{Die Empfindlichkeiten E werden gegen den Kehrwert der Beschleunigungsspannungen aufgetragen.
%                Der Plot wird mit einer linearen Ausgleichsrechnung approximiert.}
% \end{figure}
Aus der linearen Ausgleichsrechnung ergibt sich:
\begin{align}
    E(1/U_b) = a\cdot 1/U_b + b  \nonumber \\
    a = 0 && b = 0 \\ 
\end{align}
Mit der Zeichnung \ref{fig:Abmessungen} lassen sich die Maße der Röhre ablesen und a berechnen.
\begin{align}
    p = \\
    d = \\
    L = \\
    \frac{pL}{2d} = \\
    a = \\
    abs Abweichung = \\
    rel Abweichung = \\
\end{align}
