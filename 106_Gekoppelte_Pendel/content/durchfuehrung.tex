\section{Durchführung}
\label{sec:Durchfuehrung}
%Bild einfügen
\subsection{Aufbau}
    Um das gekoppelte Pendel zu untersuchen wurden zwei Pendel mit einer Aufhängung an der Wand befestigt.
    Um die Reibung der Aufhängung zu minimieren liegen die Pendel nur mit zwei Nadel auf der Aufhängung auf,da so die Angriffsfläche und somit dei Reibung minimal ist.
    Die beiden Pendel besitzen jeweils ein Loch in der Metallstange um die Pendel durch eine Feder miteinander zu koppeln.
    Die Massen an den Pendeln sind in der Höhe verstellbar.
\subsection{Versuch}
    Die Schwingungsdauern der beiden einzelnen Pendel mit der Länge 72cm wurden zuerst bestimmt.
    Dazu wurden die Pendel einmal ausgelenkt und die Dauer von 5 Schwingungen wurde mit einer Stoppuhr festgehalten.
    Diese Methode des Messens wird auch bei den weiteren Messungen verwendet.
    Nachdem für jedes Pendel diese Messung 10 mal wiederholt wurde, wurden die Schwingungsdauern für die gengensinnige und gleichsinnige Schwingung bestimmt.
    Zuletzt wurde bei der Schwebung die Schwingungsdauer und die Schwebungsdauer parallel gemessen.
    Zwischen den Messungen wurde einmal die Länge des Pendels verstellt um Werte für 72cm und für 80cm zu erhalten.