\section{Diskussion}
\label{sec:Diskussion}

In der Auswertung kam es bereits zu einigen unrealistischen und offensichtlichen
falschen Werten die alle aus Berechnungen stammten.
Im Folgenden soll die Ursache für die dramastischen Abweichungen analysiert werden.


Dafür soll zu Beginn die Messung an sich überprüft werden.
Dazu wird die jeweilige theoretische Schwingungsdauer mit den gemessenen Schwingungsdauern verglichen.
\begin{figure}
    \centering
    \includegraphics[width=0.5\textwidth]{plots/plot1.pdf}
    \caption{Gleichsinnige Schwingungen}
\end{figure}

\begin{figure}
    \begin{subfigure}[c]{0.5\textwidth}
        \includegraphics[width=\textwidth]{plots/plot2.pdf}
        \subcaption{Pedel 72cm}
    \end{subfigure}
    \begin{subfigure}[c]{0.5\textwidth}
        \includegraphics[width=\textwidth]{plots/plot3.pdf}
        \subcaption{Pendel 80cm}
        \label{subfig:pedel80}
    \end{subfigure}
    \caption{Gleichsinnige Schwingungsdauer}
\end{figure}

Dabei ergibt sich der zu erwartende theoretische Werte aus Geichung (\ref{eqn:T_m}).

Hier bei wird besonders gut deutlich, dass bei \ref{subfig:pendel80}, große Abweichungen
vorliegen. Dies lässt sich nur auf einen Messfehler zurückführen.
Weiterführend sei zu beachten, dass die Kopplungskonstante die für die theoretische Kurve
verwendet wird, ebenfalls aus diesen falschen Werten stammt.

Dies erklärt weiterführend, warum die Kopplungskonstante aus \ref{eqn:kopplung80} nicht die selben
sind, bzw. warum die Konstante resultierend aus den Schwingsungsdauern der Pendellänge 80cm so weit von der
Kopplungskonstanten die aus Schwingungsdauern des Pendel mit einer Länge von 72cm enfernt ist.

Fazit: Der Messfehler der Gegensinnigen Schwingung von der Pendellänge $l=80cm$ ist 
ist ein Ursprung der 