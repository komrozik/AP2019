\newpage
\section{Auswertung}
\label{sec:Auswertung}
\subsection{Längenprüfung}
Für die Schwingungsdauer $T$ der beiden frei schwingenden Pendel (mit gleichen Längen):

%Frei schwingende Pendel
\begin{table}
    \centering
    \label{tab:Data_freischwingend}
    \begin{tabular}{c c | c}
        \toprule
         & Pendel 1\;/\;s (72cm) & Pendel 2\;/\;s (72cm)\\
        \midrule
        & 1,688 & 1,694 \\
        & 1,668 & 1,732 \\
        & 1,688 & 1,720  \\
        & 1,700  & 1,73  \\
        & 1,674 & 1,694 \\
        & 1,688 & 1,682 \\
        & 1,712 & 1,700   \\
        & 1,694 & 1,708 \\
        & 1,70  & 1,726 \\
        & 1,686 & 1,686 \\
        \midrule
        Mittelwert $\bar{T}$ (nach Gl. \ref{eqn:mittelwert}): & 1,69 & 1,71 \\
        Standartabweichung $\sigma$ (nach Gl. \ref{eqn:standartabweichung}): & 0,012 & 0.018 \\ 
        \bottomrule
    \end{tabular} 
    \caption{Frei schwingende Pendel}
\end{table}

Es wurde dabei versucht die Längen der beiden Pedel möglichst genau gleich einzustellen.
Anhang der Mittelwerte der beiden Pendel lässt die Aussage bzw. Näherung treffen, beide Pendel seien exakt gleich lang.

Der Mittelwert bildet sich dabei aus:
\begin{equation}
    \bar{T}=\frac{1}{n}\sum_{\textrm{i=1}}^n T_\textrm{i}
    \label{eqn:mittelwert}
\end{equation}
Die Standartabweichung bildet sich aus:
\begin{equation}
    \sigma = \sqrt{\frac{\sum_{i=1}^{n}(x_i-\bar{x})^2}{n}}
    \label{eqn:standartabweichung}
\end{equation}
\newpage
%Gleichsinnig
\subsection{Gleichsinnige Schwingung}
Für die gleichsinnige Schwingung ergaben sich für die beiden Pendellängen $l=72cm$ und $l=80cm$
folgende Schwingungsdauern $T$:
\begin{table}
    \centering
    \label{tab:Data_gleichphasig}
    \begin{tabular}{c c | c}
        \toprule
        & Pendel 1\;/\;s (72cm) & Pendel 2\;/\;s (80cm)\\
        \midrule
            & 1,694 & 1,806 \\
            & 1,706 & 1,794 \\
            & 1,712 & 1,812 \\
            & 1,720 & 1,794 \\
            & 1,700 & 1,824 \\
            & 1,688 & 1,812 \\
            & 1,744 & 1,776 \\
            & 1,720 & 1,782 \\
            & 1,720 & 1,800 \\
            & 1,676 & 1,806 \\        
        \midrule
        Mittelwert $\bar{T}$ (nach Gl. \ref{eqn:mittelwert}): & 1,708 & 1,801 \\
        Standartabweichung $\sigma$ (nach Gl. \ref{eqn:standartabweichung}): & 0,019 & 0,014 \\
        \bottomrule
    \end{tabular}
    \caption{Gleichsinnige Schwingung}
\end{table}

Dabei spiegelt der Mittelwert die Proportionalität $T \sim \sqrt{l}$ (siehe Gl. \ref{eqn:T_p}) wieder.\newline
Aus den Messwerten ergibt sich eine Schwingungsfrequenz:
\begin{table}
    \centering
    \label{tab:frq_gleichs}
    \begin{tabular}{c c c}
        \toprule
        & \multicolumn{2}{c}{$w\;/\;\frac{1}{s}$}\\
        \cmidrule(lr){2-3} 
        & Pendel 72cm & Pendel 80cm\\
        \midrule
        Aus Messwerten (mit Gl.(\ref{eqn:T_p})) & 3,68 \pm 0,04 & 3,489 \pm 0,027 \\
        Berechnet (mit Gl.(\ref{eqn:w_p}))      & 3,691 & 3,501 \\
        \bottomrule
    \end{tabular}
\end{table}
\newpage

%Gegensinnig
\subsection{Gegensinnige Schwingung}
Für die gegensinnige Schwinung ergaben sich die Schwingungsdauern:
\begin{table}
    \centering
    \label{tab:Data_gegenphasig}
    \begin{tabular}{c c | c}
        \toprule
        & Pendel 1\;/\;s (72cm) & Pendel 2\;/\;s (80cm)\\
        \midrule
        & 1,574 & 1,344 \\
        & 1,568 & 1,344 \\
        & 1,582 & 1,364 \\
        & 1,576 & 1,33  \\
        & 1,588 & 1,318 \\
        & 1,574 & 1,35  \\
        & 1,568 & 1,312 \\
        & 1,588 & 1,332 \\
        & 1,556 & 1,314 \\
        & 1,564 & 1,312 \\
        \midrule
        Mittelwert $\bar{T}$ (nach Gl. \ref{eqn:mittelwert}): & 1,574 & 1,332 \\
        Standartabweichung $\sigma$ (nach Gl. \ref{eqn:standartabweichung}): & 0,01 & 0,017 \\
        \bottomrule
    \end{tabular}
    \caption{Gegensinnige Schwingung}
\end{table}
\\Dabei ist anzumerken, dass hier nach Gl. \ref{eqn:T_m} ebenfalls eine $T \sim l$ Proportionalität
gelten sollte. Dem wird sich im Abschnitt \ref{sec:Diskussion} gewidmet.\newline

Aus den Messwerten ergibt sich eine Schwingungsfrequenz:
\begin{table}
    \centering
    \label{tab:frq_gegens}
    \begin{tabular}{c c c}
        \toprule
        & \multicolumn{2}{c}{$\omega\;/\;\frac{1}{s}$}\\
        \cmidrule(lr){2-3} 
        & Pendel 72cm & Pendel 80cm\\
        \midrule
        Aus Messwerten (mit Gl.(\ref{eqn:T_m})) & 3,992\pm 0,025 & 4,72 \pm 0,06\\
        Berechnet (mit Gl.(\ref{eqn:w_m}))      & 3,722 & 3,605 \\
        \bottomrule
    \end{tabular}
\end{table}
\newpage
%Gekoppelt
\subsection{Gekoppelte Schwingung}
Für die Schwingungsdauern und Schwebungsdauern der gekoppelten Schwingung:
\begin{table}
    \centering
    \label{tab:Data_schw}
    \caption{Schwingung und Schwebung des Pendels}
    \begin{tabular}{c c | c || c | c}
        \toprule
        & \multicolumn{2}{c}{Pendel l=72cm} & \multicolumn{2}{c}{Pendel l=80cm} \\
        \cmidrule(lr){2-3}\cmidrule(lr){4-5}
        Gemessen: & $T_\text{Schwingung}$\;/\;s & $T_\text{Schwebung}$\;/\;s & $T_\text{Schwingung}$\;/\;s & $T_\text{Schwebung}$\;/\;s\\
        \midrule
        & 1,538 & 18,96 & 1,750 & 20,52 \\   
        & 1,600 & 18,67 & 1,726 & 21,00 \\
        & 1,600 & 18,43 & 1,744 & 21,38 \\
        & 1,550 & 18,76 & 1,712 & 21,43 \\
        & 1,636 & 18,95 & 1,726 & 19,83 \\
        & 1,580 & 19,38 & 1,724 & 20,93 \\
        & 1,632 & 18,46 & 1,694 & 21,84 \\
        & 1,612 & 19,21 & 1,700 & 20,76 \\
        & 1,636 & 18,66 & 1,706 & 20,50 \\
        & 1,588 & 19,09 & 1,720 & 19,70 \\
        \midrule
        Mittelwert $\bar{T}$ (nach Gl. \ref{eqn:mittelwert}): & 1,597 & 18,857 & 1,721 & 20,789 \\
        Standartabweichung $\sigma$ (nach Gl. \ref{eqn:standartabweichung}): & 0,033 & 0,299 & 0,017 &  0,647 \\
        Berechnete $T_\textrm{S}$ (nach Gl.(\ref{eqn:Schwebung})):  &       & 20,030 &       &  5,118 \\
        \bottomrule
    \end{tabular}
\end{table}
Die berechnete Schwebungsdauer der Pendellänge von 80cm, weicht dabei sehr stark ab. In Abschnitt \ref{sec:Diskussion} wird dies genauer diskutiert.\\

Aus den Messwerten ergibt sich eine Schwingungsfrequenz:
\begin{table}
    \centering
    \label{tab:frq_gleichs}
    \begin{tabular}{c c c}
        \toprule
        & \multicolumn{2}{c}{$\omega\;/\;\frac{1}{s}$}\\
        \cmidrule(lr){2-3} 
        & Pendel 72cm & Pendel 80cm\\
        \midrule
        Aus Messwerten (mit Gl.\ref{eqn:Schwebung}) & 3,93 \pm 0,08 & 3,65 \pm \\
        Berechnet (mit Gl.\ref{eqn:Schwebung}      & -0,031 & -0,103 \\
        \bottomrule
    \end{tabular}
\end{table}

\subsection{Kopplungskonstante}
Anhand der Gleichung \ref{eqn:Kopplung} lässt sich die Kopplungskonstante $K$ für
die beiden Pendellängen ermitteln:
\begin{gather}
    K \textrm{ für 72cm Pedel} = 0.0816\\
    K \textrm{ für 80cm Pedel} = 0.2926
    \label{eqn:kopplung80}
\end{gather}
(siehe dazu Abschnitt \ref{sec:Diskussion})


