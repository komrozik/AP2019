\section{Diskussion}
\label{sec:Diskussion}
\begin{table}[H]
    \centering
    \begin{tabular}{c | c c c }
        \toprule
        Messung & Wert &  absoluter Fehler & relativer Fehler\\
         & mm & mm & \\
        \midrule
        $f_1$ & 95.86 & 4.14 & 0.041\\
        $f_2$ & 48.07 & 1.93 & 0.039\\
        $f_{\text{bessel}} $& 96.97 & 3.03 & 0.030\\
        \bottomrule
    \end{tabular}
    \caption{Die berechnetetn Größen und ihre Abweichung}
    \label{tab:tab1}
\end{table}
Wie in der Tabelle erkennbar ergeben sich für die Methode mit Brennweite und Gegenstandsweite und für die Methode nach Bessel Abweichungen von maximal 4,1\%.
Man sieht auch dass die Methode nach Bessel einen genaueren Wert für die Brennweite liefert als die Berechnung mit der Gegenstandsweite und der Bildweite. \\
Bei der Methode nach Abbe ist in den beiden Graphen nur schwer ein linearer Zusammenhang zu erkennen.
Die Probleme in dieser Methode sind darauf zurückzuführen, dass die Messdaten fehlerhaft sicnd und die V Werte nicht zu den entsprechenden g' und b' Werten passen.