\section{Auswertung}
\label{sec:Auswertung}
\subsection{Bestimmung der Wellenlänge}
Wie bereits in der Theorie beschrieben lässt sich die Wellenlänge des Lasers über die Messung der Maxima auf einem Messintervall $\Delta d$ bestimmen (Formel \ref{eqn:Wellenlaenge}).
\begin{table}
    \centering
    \begin{tabular}{c c | c }
        \toprule
        $\Delta d $& Maxima & $\lambda$\\
         m &  & nm \\
        \midrule
        0.00097469 & 3030 & 643,35 \\
        0.00125972 & 3007 & 837,86 \\
        0.00099263 & 3022 & 656,94 \\
        0.00095874 & 3002 & 638,73 \\
        0.00097469 & 3013 & 646,99 \\
        \bottomrule
    \end{tabular}
    \caption{Die berechnete Wellenlänge und die Parameter von denen sie abhängt.}
    \label{tab:tab1}
\end{table}
In der Tabelle ist gut zu erkennen, dass die zweite Messung stark von den anderen abweicht und sie wird somit in den weiteren berechnungen vernachlässigt.
Um die Wellenlänge genauer zu bestimmen wird nun der Mittelwert der Messungen gebildet.
\begin{align}
    \lambda_{\text{mittel}} = 684,77 \cdot 10^{-9} \text{m} \nonumber
\end{align}
\subsection{Messung des Brechungsindex}
Nun soll der Brechungsindex von Luft bestimmt werden.
Nach Formel \ref{eqn:Brechungsindex} und \ref{eqn:Brechungsindexaenderung} gilt:
\begin{align}
    \Delta n = \frac{z \lambda}{2 b} = 1,11618 \nonumber \\
    n = 1 + \Delta n \frac{T}{T_0} \frac{p_0}{p-p'} = 1,000006742 \nonumber \\
\end{align}
Wobei die Wellenlänge aus dem vorherigen Aufgabenteil übernommen wird als ... .
