\section{Diskussion}
\label{sec:Diskussion}
Die Wellenlänge des verwendeten Lasers wurde als $ 646,5 \text{nm} $ bestimmt und liegt somit im Bereich von 632 bis 670 nm, in dem rotes Laserlicht vorkommt \cite{Wellenlaenge}.
Der berechnete Wert liegt somit im erlaubten Intervall, da im Experiment ein roter Laser verwendet wurde.\\
Der berechnete Brechungsindex von Luft $n = 1,00013$ hat einen absoluten Fehler von $\Delta_{abs}n = 0,00017$ und dem relativen Fehler $\Delta_{rel}n = 0,00017$.
Der gemessenen Brechungsindex liegt somit nah am Literaturwert von 1,0003 und die Messung war erfolgreich.