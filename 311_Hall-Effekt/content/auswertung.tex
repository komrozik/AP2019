\newpage
\section{Auswertung}
%Schonmal die Tabellen für dich ;-)

%data1
\begin{table}
    \centering
    \begin{tabular}{c c}
        \toprule
        $I_{Spule} \;/\;$A & $B\;/\;$mT\\
        \midrule
        0,5                 &64,5 \\
        1,0                 &130,3\\
        1,5                 &197,7\\
        2,0                 &272,4\\
        2,5                 &343,9\\
        3,0                 &411,4\\
        3,5                 &583,1\\
        4,0                 &548,0\\
        4,5                 &615,9\\
        5,0                 &678,3\\
        \bottomrule
    \end{tabular}
    \caption{Magnetfeldmessung parallel}
    \label{tab:Bp}
\end{table}

%data2
\begin{table}
    \centering
    \begin{tabular}{c c}
        \toprule
        $I_{Spule} \;/\;$A & $B\;/\;$mT\\
        \midrule
        0,0                 &0,0\\
        0,5                 &132,4\\
        1,0                 &270,0\\
        1,5                 &413,7\\
        2,0                 &556,4\\  
        2,5                 &698,3\\
        3,0                 &828,3\\
        3,5                 &956,0\\
        4,0                 &1063,9\\
        4,5                 &1145,2\\
        5,0                 &1204,5\\
        \bottomrule
    \end{tabular}
    \caption{Magnetfeldmessung (reihe)}
    \label{tab:Br}
\end{table}

%data3
\begin{table}
    \centering
    \begin{tabular}{c c c}
        \toprule
        $I_{Spule} \;/\;$A & $U_H\;/\;$mV & $I_{durch} \;/\;$A\\
        \midrule
        0                   &0,0081              &10\\
        0,5                 &0,0055              &10\\
        1,0                 &0,0039              &10\\
        1,5                 &0,0012              &10\\
        2,0                 &-0,0008             &10\\
        2,5                 &-0,0025             &10\\
        3,0                 &-0,0048             &10\\
        3,5                 &-0,0068             &10\\
        4,0                 &-0,0086             &10\\
        4,5                 &-0,0097             &10\\
        5,0                 &-0,0110             &10\\
        \bottomrule
    \end{tabular}
    \caption{Hall Spannung für Kupfer- konstanter Duchflussstrom}
    \label{tab:Cu_I}
\end{table}

%data4
\begin{table}
    \centering
    \begin{tabular}{c c c}
        \toprule
        $I_{Spule} \;/\;$A & $U_H\;/\;$mV & $I_{durch} \;/\;$A\\
        \midrule
            5                   & 0,0029&             0\\
            5                   & 0,0005&             1\\
            5                   &-0,0015&             2\\
            5                   &-0,0030&             3\\
            5                   &-0,0044&             4\\
            5                   &-0,0056&             5\\
            5                   &-0,0075&             6\\
            5                   &-0,0088&             7\\
            5                   &-0,0110&             8\\
            5                   &-0,0129&             9\\
            5                   &-0,0140&             10\\
        \bottomrule
    \end{tabular}
    \caption{Hall Spannung für Kupfer- konstantes Magnetfeld}
    \label{tab:Cu_B}
\end{table}

%data5
\begin{table}
    \centering
    \begin{tabular}{c c c}
        \toprule
        $I_{Spule} \;/\;$A & $U_H\;/\;$mV & $I_{durch} \;/\;$A\\
        \midrule
            0                   &-0,4094&             10\\
            0,5                 &-0,4042&             10\\
            1,0                 &-0,4024&             10\\
            1,5                 &-0,4004&             10\\
            2,0                 &-0,3981&             10\\
            2,5                 &-0,3981&             10\\
            3,0                 &-0,3950&             10\\
            3,5                 &-0,3928&             10\\
            4,0                 &-0,3900&             10\\
            4,5                 &-0,3880&             10\\
            5,0                 &-0,3800&             10\\
        \bottomrule
    \end{tabular}
    \caption{Hall Spannung für Zink- konstanter Durchflussstrom}
    \label{tab:Zn_I}
\end{table}

%data6
\begin{table}
    \centering
    \begin{tabular}{c c c}
        \toprule
        $I_{Spule} \;/\;$A & $U_H\;/\;$mV & $I_{durch} \;/\;$A\\
        \midrule
            5                   &-0,0012&             0\\
            5                   &-0,0373&             1\\
            5                   &-0,0755&             2\\
            5                   &-0,1136&             3\\
            5                   &-0,1530&             4\\
            5                   &-0,1919&             5\\
            5                   &-0,2291&             6\\
            5                   &-0,2671&             7\\
            5                   &-0,3090&             8\\
            5                   &-0,3512&             9\\
            5                   &-0,3957&             10\\
        \bottomrule
    \end{tabular}
    \caption{Hall Spannung für Zink- konstantes Magnetfeld}
    \label{tab:Zn_B}
\end{table}

%data7
\begin{table}
    \centering
    \begin{tabular}{c c c}
        \toprule
        $I_{Spule} \;/\;$A & $U_H\;/\;$mV & $I_{durch} \;/\;$A\\
        \midrule
            0                   &-0,1707&             10\\
            0,5                 &-0,1740&             10\\
            1,0                 &-0,1777&             10\\
            1,5                 &-0,1809&             10\\
            2,0                 &-0,1850&             10\\
            2,5                 &-0,1885&             10\\
            3,0                 &-0,1909&             10\\
            3,5                 &-0,1945&             10\\
            4,0                 &-0,1970&             10\\
            4,5                 &-0,1990&             10\\
            5,0                 &-0,2002&             10\\
       \bottomrule
    \end{tabular}
    \caption{Hall Spannung für Silber- konstanter Durchflussstrom}
    \label{tab:Ag_I}
\end{table}

%data8
\begin{table}
    \centering
    \begin{tabular}{c c c}
        \toprule
        $I_{Spule} \;/\;$A & $U_H\;/\;$mV & $I_{durch} \;/\;$A\\
        \midrule
  5                   &0,0005&              0\\
  5                   &-0,0207&             1\\
  5                   &-0,0394&             2\\
  5                   &-0,0599&             3\\
  5                   &-0,0794&             4\\
  5                   &-0,0991&             5\\
  5                   &-0,1185&             6\\
  5                   &-0,1394&             7\\
  5                   &-0,1602&             8\\
  5                   &-0,1821&             9\\
  5                   &-0,2026&             10\\

       \bottomrule
    \end{tabular}
    \caption{Hall Spannung für Silber- konstantes Magnetfeld}
    \label{tab:Ag_I}
\end{table}


\label{sec:Auswertung}
