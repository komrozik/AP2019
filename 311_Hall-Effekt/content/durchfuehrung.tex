\newpage
\section{Durchführung}
\label{sec:Durchfuehrung}
Um passende Werte für das Magnetfeld $\vec{B}$ zu erhalten muss zu Beginn des Versuchs
mit einer Hall Sonde das Magnetfeld in Abhängigkeit des Spulenstroms $I_{Spule}$ gemessen werden.\newline
Nachdem das Magnetfeld gemessen wurde können die Proben in die Halterung eingespannt werden.
Sie beeinhalten ein dünnes Plätchen der Materialien Kupfer, Silber, Zink.\\
Bei konstantem Durchflussstrom $I_{Spule}$ wird nun der Spulenstrom von 0 Ampere bis auf 5 Ampere erhöht und die Hall Spannung $U_H$ 
des Material gemessen.\\
Anschließend wird der Spulenstrom konstant auf 5 Ampere eingestellt und die Hall Spannung in Abhängigkeit von dem Durchflussstrom gemessen.
Die Messungen der Hall Spannung werden für jedes der 3 Materialien wiederholt (Kupfer,Silber,Zink).