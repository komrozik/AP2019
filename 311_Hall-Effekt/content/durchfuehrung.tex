\newpage
\section{Durchführung}
\label{sec:Durchfuehrung}
Um passende Werte für das Magnetfeld zu erhalten muss zu Beginn des Versuchs
 mit einer Hall Sonde das Magnetfeld in Abhängigkeit des Spulenstroms gemessen werden.

Nachdem das Magnetfeld gemessen wurde können die Proben in die Halterung eingespannt werden.
Bei konstantem Durchflussstrom wird nun der Spulenstrom von 0 Ampere bis auf 5 Ampere erhöht und die Hall Spannung durch das Material gemessen.
Anschließend wird der Spulenstrom konstant auf 5 Ampere eingestellt und die Hall Spannung in Abhängigkeit von dem Durchflussstrom gemessen.
Die Messungen der Hall Spannung werden für jedes der 3 Materialien wiederholt(Kupfer,Silber,Zink).
Außerdem wird die Dicke der Metallplatten mithilfe einer Mikrometerschraube gemessen und zur berechnung der spezifischen Leitfähigkeit wird der Wiederstand eines Metalldrahts bekannter Länge gemessen.