\section{Diskussion}
Durch die gemessenen Dicken der Platten und des Drahts weichen die berechneten spezifischen Wiederstände ab von den Literaturwerten(\cite{Literaturwerte_Spezifische_Wiederstände}).
Bei Kupfer weicht die berechnetet Leitfähigkeit $\rho = 0,074$ um 311 \% vom Literaturwert ab, bei Silber weicht das berechnete $\rho = 0,0127$ um 20\% vom Literaturwert ab.
Da die Werte für die weiteren Berechnungen weiter verwendet werden, haben wir uns aufgrund der großen Fehler entschlossen dass wir mit den Literaturwerten weiterrechnen.
Die gemessenen Werte für unsere Materialkonstanten liegen in den richtigen Größenordnungen, allerdings fällt auch hier auf, dass die Werte von den Literaturwerten abweichen.\\
Die Hall Konstante $A_H = \frac{1}{n\cdot e_0}$ wird für Kupfer einmal als $\SI{-2.79\pm 0.08 e-11}{\cubic \metre \per \coulomb}$ und als $\SI{-2.30\pm0.05 e-11}{\cubic \metre \per \coulomb}$ bestimmt, beide Werte liegen zwar in der richtigen Größenordnung, weichen jedoch stark vom Literaturwert $A_H = \SI{-5.30 e-11}{\cubic \metre \per \coulomb}$ ab.
Für Silber liegen die Hall Konstanten bei  $\SI{-2.45 \pm 0.09 e-10}{\cubic \metre \per \coulomb}$ und $\SI{-6.124\pm 0.033 e-08}{\cubic \metre \per \coulomb}$, also auch mit großem Abstand zum Literaturwert $\SI{-9,0 e-11}{\cubic \metre \per \coulomb}$.
Die Werte für Zink, $\SI{-1.94\pm 0.15 e-10}{\cubic \metre \per \coulomb}$ und $\SI{-3.042\pm0.025 e-09}{\cubic \metre \per \coulomb}$, liegen auch in der richtigen Größenordnung aber haben große Abweichung zum Literaturwert $\SI{6,4 e-11}{\cubic \metre \per \coulomb}$.
Anhand der großen Abweichungen ist zu vermuten, dass Fehler in den Messungen enthalten sind, somit sind die berechneten Werten nicht aussagekräftig.
\label{sec:Diskussion}
