\newpage
\section{Theorie}
\label{sec:theorie}


\subsection*{Ziel}
Mit dem Versuch sollen verschiedene mikroskopische Paramteter verschiedener Metallproben 
(Kupfer, Silber und Zink) bestimmt und ausgewertet werden.
Darunter zählen:\\
\begin{itemize}\itemsep0pt
    \item Spezifischer Widerstand $\rho$
    \item Spezifische Leifähigkeit $\sigma$
    \item Ladungsträger pro Volumen
    \item Zahl der Ladungsträger pro Atom
    \item Mittlere Flugzeit
    \item Mittlere Dirftgeschwindigkeit $\bar{v_d}$
    \item Beweglichkeit $\mu$
    \item Totalgeschwindigkeit
    \item Mittlere freie Weglänge $\bar{l}$
\end{itemize}
\subsection{Bandstruktur und elektrische Leitfähigkeit von Kristallen}
Atome sind eng benachbart. Aus diesem Grund folgen die Elektronen dem Pauli-Prinzip.
D.h. innerhalb eines System dürfen die Elektronen nicht im gleichen Quantenzustand vorliegen.
Somit besitzt jedes Elektron eine geringfügig unterschiedliche Energie. Die Energiebänder gehen dabei
aus diesen Energieniveaus hervor.\\
\begin{description}
\item[Energiebänder überlappen sich]
Elektronen können in einem spontanen Übergang vom unteren in das obere Band übergehen. Sie bilden dabei Leerstellen,
welche als "Löcher" bezeichnet werden. Beim Einfluss eines äußeren elektrischen Feldes verhalten sich diese wie
positive Ladungen und sind somit ortsveränderlich.\\
Dies hat Einfluss auf die elektrische Leitfähigkeit. Das Resultat bilder der anomale Hall-Effekt. 
\item[Energiebänder bilden Löcher]
Tritt zwischen zwei Bändern eine endliche Lücke auf, so nennt man diese "verbotene Zone".
Diese beeinhalten Energien, welche die Elektronen im Festkörper nicht annehmen können.
\end{description}
Sind die Atomschalen komplett gefüllt, so können die Elektronen keine Energien aufnehmen und
abgeben, folglich tragen sie nicht zur Leitfähigkeit bei.
Ist die äußere Schale nicht komplett gefüllt, so können die Elektronen Energien aufnehmen/abgeben.
Legt man nun ein elektrisches Feld an, so nimmt das Elektron Energie auf und bewegt sich auf den elektrischen
Feldlinien, ein makroskopischer Strom ist messbar. Diese Elektronen nennt man Leistungselektronen.\\

Generell wechselwirken die Leistungselektronen nicht miteinander und den zurückgelassenden Ionenrümpfen.
Idealisiert hat das MAterial somit eine unedlich hohe Leifähigkeit.\\
Praktisch führen Kristallbaufehler und Verschmutzungen zu Zusammenstößen zwischen den bewegten Elektronen und den Fehlstellen.
Die mittlere Flugzeit $\vec{\tau}$ spiegelt dabei das Zeitintervall zwischen zwei Zusammenstößen wieder.\\
Wird nun ein elektrische Feld an eine Probe angelegt, so bewegt sich das Elektron gleichmäßig beschleunigt in der Zeit $\bar{\tau}$
entlang von $\vec{E}$.\\
Die Beschleunigung $\vec{b}$ ist abhängig von der Ladung des Elektrons (also $-e_0$) und dessen Ruhemasse $m_0$ 
\begin{equation}
    \vec{b}=-\frac{e_0}{m_0}\vec{E}
\end{equation}
Daraus resultier die Geschwindigkeitsänderung $\Delta \vec{\bar{v}}$
\begin{equation}
    \Delta \vec{\bar{v}}=-\frac{e_0}{m_0}\vec{E}\bar{\tau}
\end{equation}
Für die Driftgeschwindigkeit $\vec{\bar{v_d}}$ gilt:
\begin{equation}
    \vec{\bar{v_d}}=\frac{1}{2} \Delta \vec{\bar{v}}=\mu\vec{E}
\end{equation}
mit der Beweglichkeit $\mu$

\subsection{Elektrischer Widerstand homogener Leiter}
Die Leitfähigkeit $S$:
\begin{equation}
    S=\frac{1}{2}\frac{e_0^2}{m_0}n\bar{\tau}\frac{Q}{L}
\end{equation}

Der Strom $I$ ist:
\begin{equation}
    I=S \cdot U
\end{equation}
Der elektrische Widerstand $R$ ergibt sich somit aus:
\begin{equation}
    R=2\frac{m_0}{e_0^2}\frac{1}{n\bar{\tau}}\frac{L}{Q}=\frac{1}{S}
\end{equation}

Daraus folgende die geometrieunabhängigen Größen:\\
spezifische Leifähigkeit $\sigma$
\begin{equation}
    \sigma = \frac{1}{2}\frac{e_0^2}{m_0}n\bar{\tau}
\end{equation}
Spezifischer Widerstand $\rho$
\begin{equation}
    \rho = 2 \frac{m_0}{e_0^2}\frac{1}{n\bar{\tau}}
\end{equation}



\subsection{Der Hall Effekt - Bestimmung von Ladungsträgerdichte n}
Für die enstehende Spannung (genannt Hallspannung ($U_H$) bei einer stromdurchflossenden $I_q$
elektrischen Leiterplatte der dicke $b$ im Magnetfeld $\vec{B}$ gilt
\begin{equation}
    U_H=E_y \cdot b = \bar{v_d}\;B\cdot b = -\frac{1}{n \; e_0}\frac{B \cdot I_q}{d}
    \label{eqn:U_Hall}
\end{equation}
\begin{figure}[H]
    \centering
    \includegraphics[width=0.7\textwidth]{bilder/HAll.jpg}
    \caption{Anordungung zur Beobachtung des Hall Effektes \cite[265]{Anleitung}}
    \label{fig:hall}
\end{figure}



\subsection{Leitfähigkeitsparameter}
Mittlere freie Weglänge 
\begin{equation}
    \bar{l}=\bar{\tau} \cdot |v| \approx \bar{\tau}\sqrt{\frac{2\;E_F}{m_0}}  
\end{equation}
mit der Totalgeschwindigkeit $|v|$ bei der Fermi-Energie $E_F$
\begin{equation}
    |v|\approx \sqrt{\frac{2\;E_F}{m_0}}    
\end{equation} 






