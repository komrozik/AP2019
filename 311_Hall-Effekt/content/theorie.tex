\newpage
\section{Theorie}
\label{sec:theorie}
\subsection{Ziele}
Mit dem Versuch sollen verschiedene mikroskopische Paramteter verschiedener Metallproben bestimmt und ausgewertet werden.
\subsection{Elektronen pro Volumeneinheit und mittlere Flugzeit}
Als mittlere Flugzeit $\tau querstrich$ wird die Zeit bezeichnet die ein Elektron zwischen zwei Zusammenstößen im Kristallgitter des Metalls "fliegt".
Das Elektron wird dabei durch ein außen anliegendes E-Feld beschleunigt
\begin{equation*}
     b = - \frac{e_0}{m_0} E %Vektorpfeile!!!!!!!!
\end{equation*} 
Damit ergibt sich eine mittlere Geschwindigkeitsänderung von 
\begin{equation*}
    \Delta v = -\frac{e_0}{m_0} E \tau %Vektorpfeile!!!!!!!!
\end{equation*}
und die mittlere Driftgeschwindigkeit 
\begin{equation*}
     v_d = \frac{1}{2}\Delta v %Vektorpfeile!!!!!!!!
\end{equation*}
mit $n$ Elektronen pro Volumeneinheit kann die Stromdichte $j$ berechnet werden mit
\begin{equation*}
    j = -n v_d e_0 = \frac{1}{2} \frac{e_0^2}{m_0} E \tau n%Vektorpfeile!!!!!!!!
\end{equation*}

