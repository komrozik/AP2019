\section{Ziel}
Ziel ist es die elastische Konstante Torsionsmodul $G$ einer Metallegierung, sowie das magnetische Moment eines Permanentmagneten zu bestimmen.


\section{Theorie}
\subsection{Elastizitätstheorie}
Elastische Materialkonstanten beschreiben, die aus angreifende Kräfte pro Flächeneinheit resultierenden Deformationen und Volumenänderungen.
Dabei unterscheidet man in der Mechanik, unter Kräften, die an jedem Volumenelement angreifen, als Resultat von Translation bzw. Rotationen und
den Kräften die an jedem Oberflächenelement aufgrund von Gestalts- und Volumenänderungen angreifen.
Wirkt die Kraft senkrecht zu Oberfläche so spricht man von Normalspannung $\sigma$ bzw. Druck $p$ oder parallel zur Oberfläche,
dann spricht man von der Tangential- bzw Schubspannung $\tau$.\\




\subsection{Elastische Konstanten isotroper Stoffe}
In der Praxis haben sich für isotrope Stoffe vier Elastische Konstanten durchgesetzt.\\
\begin{description}
\item[Elastizitätsmodul $E$:]


\item[Schubmodul,Torsionsmodul $G$:]
\item[Kompressionsmodul $Q$:]
\item[Poissonsche Querkontraktionszahl $\mu$:]
\end{description}


Kehrt das Material nach der Einwirkung der Kräft wieder in seine ursprüngliche Form zurück so, spricht man von einer elastischen Deformation.
Dabei ist die Verformung proportional zu einwirkenden Belastung durch kleine Spannungen (Hooksche Gesetz).
\begin{align}
    \sigma &= E  \frac{\Delta L}{L} & \mathrm{bzw.} & P=Q \frac{\Delta V}{V} 
\end{align}

\subsection{Schubmodul/Torsionmodul G}
\subsection{Das magnetische Moment des Permanentmagneten}
\label{sec:Theorie}
