\section{Ziel}
Ziel ist es die elastische Konstante Torsionsmodul $G$ einer Metallegierung sowie das magnetische Moment eines Permanentmagneten zu bestimmen.


\section{Theorie}
\subsection{Elastizitätstheorie}
Elastische Materialkonstanten beschreiben die aus angreifende Kräfte pro Flächeneinheit resultierenden Deformationen und Volumenänderungen.
Dabei unterscheidet man in der Mechanik, unter Kräften, die an jedem Volumenelement angreifen, als Resultat von Translation bzw. Rotationen und
den Kräften die an jedem Oberflächenelement aufgrund von Gestalts- und Volumenänderungen angreifen.



\subsection{Elastische Konstanten isotroper Stoffe}
In der Praxis haben sich für isotrope Stoffe vier Elastische Konstanten durchgesetzt.\\
\begin{description}
\item[Elastizitätsmodul $E$:]
\item[Schubmodul,Torsionsmodul $G$:]
\item[Kompressionsmodul $Q$:]
\item[Poissonsche Querkontraktionszahl $\mu$:]
\end{description}

\subsection{Schubmodul/Torsionmodul G}
\subsection{Das magnetische Moment des Permanentmagneten}
\label{sec:Theorie}
