\section{Theorie}
\label{sec:Theorie}
Der Versuch besteht aus 3 kleineren Versuchen, von denen 2 dazu dienen die elastischen Konstanten einer Probe zu bestimmen und ein Versuch dazu da ist um den magnetischen Moment eines Permanentmagneten zu bestimmen.
Die Materialeigenschaften eines Stoffes im Bezug auf die Verformung durch äußere Kräfte werden durch verschiedene materialspezifische Konstanten beschrieben, die im folgenden nun kurz erläutert werden.
\begin{description}
    \item[Schubmodul $G$]
        Das Schubmodul beschreibt die Veränderung der Gestalt des Materials durch angreifende Kräfte.
    \item[Elastizitätsmodul $E$]
    \item[Poisson'sche Querkontraktionszahl $\mu$]
    \item[Kompressionsmodul $Q$]
\end{description}