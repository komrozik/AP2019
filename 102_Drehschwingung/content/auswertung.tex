%TO DO
%Trägheitsmoment Einheiten?
%Fehler erklären?

\newpage
\section{Auswertung}
Zu Beginn, muss das Gesamtträgheitsmoment $\theta_\text{Ges}$ ermittelt werden.
Dies ergibt sich aus dem Trägheitsmoment der Kugel $\theta_\text{Kugel}$ (aus Gleichung \ref{eqn:trägheitsmoment}) und
dem Trägheitsmoment der Kugelhalterung $\theta_\text{Halterung}$.
\begin{gather}
    \theta_\text{Kugel}=0.131972 \pm 0.000018\;\mathrm{?}\\
    \theta_\text{Halterung}=22,5 \;\mathrm{gcm^2}\\ %umrechnung in plot.py in was?
    \theta_\text{Ges}=0.131975 \pm 0.000018 \;\mathrm{?}
\end{gather}

\subsection{Das Schubmodul $G$ - Messungen ohne B-Feld}
\begin{table}
    \centering
    \label{tab:tabelle_1}
    \begin{tabular}{p{3cm} | p{2cm}}
    \toprule
    & T(s) \\
    \midrule
    &18.696\\
    &18.699\\
    &18.702\\
    &18.763\\
    &18.699\\
    &18.695\\
    &18.591\\
    &18.647\\
    &18.673\\
    &18.665\\
    &18.663\\
    &18.635\\
    \midrule
    Mittelwert:     & ? \\
    Ungenauigkeit:     & ? \\
    \bottomrule
    \end{tabular}
    \caption{Periodendauer T}
\end{table}
\newpage
\subsubsection{Berechnung des Schubmoduls $G$}
Benötigt werden folgende Werte, um das Schubmodul $G$ nach Gleichung \ref{eqn:schubmodul_formel}
zu bestimmen.
\begin{table}
    \centering
    \label{tab:tabelle_1}
    \begin{tabular}{p{3cm} | p{1.5cm} | p{1.5cm} | p{1.5cm} | p{1.5cm}}
    \toprule
    & L\;/\;cm & d \;/\;mm & $m_\text{k}$\;/\;g & 2$R_\text{k}$\;/\;mm\\
    \midrule
    & 66,38 & 0,19 & 512.2 &  50.76 \\
    &      & 0,19 &       &        \\    
    &      & 0,2  &       &        \\
    &      & 0,2  &       &        \\
    \midrule
    Mittelwert:    &      & 0,0194 &       &        \\
    Ungenauigkeit: & 0,07 & 0,0004 & 0,205 & 0,0036 \\
    \bottomrule
    \end{tabular}
    \caption{Messunggrößen der Apperatur}
    \label{tab:tabelle_2}
\end{table}

Für das Schubmodul $G$ ergibt sich somit:
\begin{equation*}
    G = (7,3 \cdot 10^{10} \pm 0,7 \cdot 10^{10} )\mathrm{\frac{N}{m^2}}
\end{equation*}
%G = (7.3\pm 0.7)e+10
Dies enspricht einem prozentualen Fehler von $9,6\%$.

\subsubsection{Berechnung der Querkontraktionszahl}
Mit der Gleichung \ref{eqn:Beziehung} ergibt sich für die Poissonsche Querkontraktionszahl $\mu$
\begin{equation*}
    \mu = (0,45 \pm 0,13) \mathrm{?}
    %0.45+/-0.13
\end{equation*}
Dies entspricht einem prozentualen Fehler von $28,9\%$.

\subsubsection{Berechnung des Kompressionsmoduls Q}
Mit der Gleichung \ref{eqn:Beziehung} 
folgt für das Kompressionsmodul $Q$
\begin{equation*}
    Q = (0,7 \pm 1,7) 10^{12} \;\mathrm{Pa}
    %(0.7+/-1.7)e+12
\end{equation*}
Dies entspricht einem prozentualen Fehler von $242,8\%$.
\newpage
\subsection{Das Magnetische Moment - Messung mit B-Feld}

Für das Magnetische Feld $B$ im Inneren einer Helmholtz-Spulen gilt:
\begin{equation}
    B= \frac{8}{\sqrt{125}} \cdot \frac{\mu_0 I N}{R}
\end{equation}
wobei $N=390$ die Windungszahl und $R=78$mm der Radius der Spulen ist.\\
Aus Gleichung \ref{eqn:periode}
folgt:
\begin{equation}
    \to mB+D := \zeta = \frac{4\pi^2\theta_\text{Ges}}{T^2_\text{m}}
\end{equation}

Damit ist $\zeta$ eine lineare Funktion. Daraus folgen dann die Parameter $m$, $D$ und $B$.
Stellt man nun $\zeta$ um, so ergibt sich das Magnetische Moment $m_\text{mag}$:
%HIER PLOT EINFÜGEN
\begin{equation}
    m_\text{mag}=\frac{\zeta}{}    
\end{equation}

\begin{gather}
    m = \\
    D = \\
\end{gather}

\newpage
\begin{table}
    \centering
    \label{tab:tabelle_1}
    \begin{tabular}{p{3cm} | p{1.5cm} p{1.5cm} p{1.5cm} p{1.5cm} p{1.5cm}}
      &      &      & T(s) &      &     \\
    I & 0,1A & 0,2A & 0,3A & 0,4A & 0,5A\\
    \midrule
    & 13.168 & 10.583 & 8.871 &  7.444 &  6.639\\   
    & 13.156 & 10.577 & 8.873 &  7.444 &  6.856\\   
    & 13.151 & 10.564 & 8.869 &  7.465 &  6.840\\   
    & 13.140 & 10.565 & 8.864 &  7.436 &  6.849\\   
    & 13.130 & 10.558 & 8.860 &  7.440 &  6.860\\   
    & 13.103 & 10.552 & 8.855 &  7.462 &  6.855\\   
    & 13.069 & 10.530 & 8.850 &  7.440 &  6.850\\   
    & 13.085 & 10.527 & 8.842 & 7.463  &  6.842\\   
    & 13.074 & 10.518 & 8.835 &  7.445 &  6.835\\   
    & 13.011 & 9.773  & 8.823 &  7.428 &  6.823\\ 
    \midrule
    Mittelwert:    & 13.109 & 10.475 & 8.854 & 7.447 & 6.825 \\
    \midrule
    B-Feld : & & & & & \\
    $\zeta$: & & & & & \\
    \bottomrule
    \end{tabular}
    \caption{Periodendauer T des Drahtes mit B-Feld von 0,1A - 0,5A}
    \label{tab:tabelle_01A}
\end{table}

\begin{table}
    \centering
    \label{tab:tabelle_1}
    \begin{tabular}{p{3cm} | p{1.5cm} p{1.5cm} p{1.5cm} p{1.5cm} p{1.5cm}}
      &      &      & T(s) &      &     \\
    I & 0,6A & 0,7A & 0,8A & 0,9A & 1,0A\\
    \midrule
    & 4.722 &  5.777 &  5.384 &  5.018 &  4.733\\
    & 6.237 &  5.767 &  5.378 &  5.029 &  4.779\\
    & 6.261 &  5.784 &  5.376 &  5.077 &  4.747\\
    & 6.227 &  5.767 &  5.362 &  5.045 &  4.765\\
    & 6.208 &  5.795 &  5.376 &  5.008 &  4.762\\
    & 6.229 &  5.762 &  5.363 &  5.038 &  4.728\\
    & 6.227 &  5.774 &  5.368 &  5.052 &  4.777\\
    & 6.178 &  5.784 &  5.359 &  5.061 &  4.756\\
    & 6.183 &  5.761 &  5.364 &  5.034 &  4.792\\
    & 6.163 &  5.791 &  5.373 &  5.028 &  4.789\\
    \midrule
    Mittelwert:    & 6.064 & 5.776 & 5.370 & 5.039 &  4.763 \\
    \midrule
    B-Feld : & & & & & \\
    $\zeta$: & & & & & \\
    \bottomrule
    \end{tabular}
    \caption{Periodendauer T des Drahtes mit B-Feld von 0,6A - 1,0A}
    \label{tab:tabelle_06A}
\end{table}

\label{sec:Auswertung}
