%TO DO
%Trägheitsmoment Einheiten?
%Fehler erklären?

\newpage
\section{Auswertung}
Zu Beginn, muss das Gesamtträgheitsmoment $\theta_\text{Ges}$ ermittelt werden.
Dies ergibt sich aus dem Trägheitsmoment der Kugel $\theta_\text{Kugel}$ (aus Gleichung \ref{eqn:trägheitsmoment}) und
dem Trägheitsmoment der Kugelhalterung $\theta_\text{Halterung}$.
\begin{gather}
    \theta_\text{Kugel}=0.131972 \pm 0.000018\;\mathrm{?}\\
    \theta_\text{Halterung}=22,5 \;\mathrm{gcm^2}\\ %umrechnung in plot.py in was?
    \theta_\text{Ges}=0.131975 \pm 0.000018 \;\mathrm{?}
\end{gather}

\subsection{Das Schubmodul $G$ - Messungen ohne B-Feld}
\begin{table}
    \centering
    \label{tab:tabelle_1}
    \begin{tabular}{p{3cm} | p{2cm}}
    \toprule
    & T(s) \\
    \midrule
    &18.696\\
    &18.699\\
    &18.702\\
    &18.763\\
    &18.699\\
    &18.695\\
    &18.591\\
    &18.647\\
    &18.673\\
    &18.665\\
    &18.663\\
    &18.635\\
    \midrule
    Mittelwert:     & ? \\
    Ungenauigkeit:     & ? \\
    \bottomrule
    \end{tabular}
    \caption{Periodendauer T}
\end{table}
\newpage
\subsubsection{Berechnung des Schubmoduls $G$}
Benötigt werden folgende Werte, um das Schubmodul $G$ nach Gleichung \ref{eqn:schubmodul_formel}
zu bestimmen.
\begin{table}
    \centering
    \label{tab:tabelle_1}
    \begin{tabular}{p{3cm} | p{1.5cm} | p{1.5cm} | p{1.5cm} | p{1.5cm}}
    \toprule
    & L\;/\;cm & d \;/\;mm & $m_\text{k}$\;/\;g & 2$R_\text{k}$\;/\;mm\\
    \midrule
    & 66,38 & 0,19 & 512.2 &  50.76 \\
    &      & 0,19 &       &        \\    
    &      & 0,2  &       &        \\
    &      & 0,2  &       &        \\
    \midrule
    Mittelwert:    &      & 0,0194 &       &        \\
    Ungenauigkeit: & 0,07 & 0,0004 & 0,205 & 0,0036 \\
    \bottomrule
    \end{tabular}
    \caption{Messunggrößen der Apperatur}
    \label{tab:tabelle_2}
\end{table}

Für das Schubmodul $G$ ergibt sich somit:
\begin{equation*}
    G = (7,3 \cdot 10^{10} \pm 0,7 \cdot 10^{10} )\mathrm{\frac{N}{m^2}}
\end{equation*}
%G = (7.3\pm 0.7)e+10
Dies enspricht einem prozentualen Fehler von $9,6\%$.

\subsubsection{Berechnung der Querkontraktionszahl}
Mit der Gleichung \ref{eqn:Beziehung} ergibt sich für die Poissonsche Querkontraktionszahl $\mu$
\begin{equation*}
    \mu = (0,45 \pm 0,13) \mathrm{?}
    %0.45+/-0.13
\end{equation*}
Dies entspricht einem prozentualen Fehler von $28,9\%$.

\subsubsection{Berechnung des Kompressionsmoduls Q}
Mit der Gleichung \ref{eqn:Beziehung} 
folgt für das Kompressionsmodul $Q$
\begin{equation*}
    Q = (0,7 \pm 1,7) 10^{12} \;\mathrm{Pa}
    %(0.7+/-1.7)e+12
\end{equation*}
Dies entspricht einem prozentualen Fehler von $242,8\%$.
\newpage
\subsection{Das Magnetische Moment - Messung mit B-Feld}

Für das Magnetische Feld $B$ im Inneren einer Helmholtz-Spulen gilt:
\begin{equation}
    B= \frac{8}{\sqrt{125}} \cdot \frac{\mu_0 I N}{R}
\end{equation}
wobei $N=390$ die Windungszahl und $R=78$mm der Radius der Spulen ist.


\label{sec:Auswertung}
