\newpage
\section{Diskussion}
\label{sec:Diskussion}
Die Ungenauigkeitein in den gemessenen Werten sind durch das Pendeln der Messapparatur zu erklären, da schon kleine Störungen die Kugel zum Pendeln anregten und somit die Messung verfälschten.
Außerdem traten bei der Messung des magnetischen Moments Fehler auf, weil es mühsam und ungenau war die Stellschraube so zu bewegen, dass der Spiegel den Lichtstrahl beim pendeln auch über die Photodioden lenkt und nicht nur daneben pendelt.
Messabweichungen die deutlich von den anderen 9 Werten der Messung abwichen wurden aus den Berechnungen gestrichen.
Fehler durch das stoppen der Uhr konnten in diesem Versuch vernachlässigt werden, da ducrch die Quarzuhr und die Schaltung sehr genaue Messweerte entstsanden.
Die Ergebnisse des ersten versuchs scheinen realistisch, mit der ausnahme des Wertes für Q? dessen Fehler viel zu groß ist.
Der Fehler von Q kann durch eine ungenaue Ablesung und große Fehlerfortpflanzung erklärt werden.
Die Werte für das magnetische Moment scheinen auch realistisch.