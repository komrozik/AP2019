\newpage
\section{Diskussion}
\label{sec:Diskussion}
Die Ungenauigkeiten in den gemessenen Werten sind durch das Pendeln der Messapparatur zu erklären,
 da schon kleine Störungen die Kugel zum Pendeln anregten und somit die Messung verfälschten.
Außerdem traten bei der Messung des magnetischen Moments Fehler auf,
 weil es mühsam und ungenau war die Stellschraube so zu bewegen,
  dass der Spiegel den Lichtstrahl beim pendeln auch über die Photodiode lenkt und nicht nur daneben pendelt.
Messabweichungen die deutlich von den anderen Werten der Messung abwichen wurden aus den Berechnungen gestrichen.
Fehler durch das stoppen der Uhr konnten in diesem Versuch vernachlässigt werden,
 da durch die Quarzuhr und die Schaltung sehr genaue Messwerte entstehen.
Die Ergebnisse des ersten versuchs scheinen realistisch, mit der Ausnahme des Wertes für Q dessen Fehler viel zu groß ist.
Der Fehler von Q kann durch eine ungenaue Ablesung und große Fehlerfortpflanzung erklärt werden.
Die Werte für das magnetische Moment scheinen auch realistisch.

Vergleicht man die Module mit den Literaturwerten von Stahl:
\begin{equation}
    $E_{\textrm{Lit}}=(21 \pm 0.05)10^{10} \textrm{\frac{N}{m^2}}$\\
    $G_{\textrm{Lit}}=$\\
    $Q_{\textrm{Lit}}=$\\
\end{equation}