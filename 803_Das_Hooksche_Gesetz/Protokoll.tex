\documentclass[titlepage = firstcover]{scrartcl}
\usepackage[aux]{rerunfilecheck}
\usepackage{fontspec}
\usepackage[main=ngerman, english, french]{babel}

% mehr Pakete hier
\usepackage{expl3}
\usepackage{xparse}

%Mathematik------------------------------------------------------
\usepackage{amsmath}   % unverzichtbare Mathe-Befehle
\usepackage{amssymb}   % viele Mathe-Symbole
\usepackage{mathtools} % Erweiterungen für amsmath
\usepackage[
  math-style=ISO,    % \
  bold-style=ISO,    % |
  sans-style=italic, % | ISO-Standard folgen
  nabla=upright,     % |
  partial=upright,   % /
]{unicode-math}% "Does exactly what it says on the tin."

% Laden von OTF-Mathefonts
% Ermöglich Unicode Eingabe von Zeichen: α statt \alpha

\setmathfont{Latin Modern Math}
%\setmathfont{Tex Gyre Pagella Math} % alternativ zu Latin Modern Math
\setmathfont{XITS Math}[range={scr, bfscr}]
\setmathfont{XITS Math}[range={cal, bfcal}, StylisticSet=1]

\AtBeginDocument{ % wird bei \begin{document}
  % werden sonst wieder von unicode-math überschrieben
  \RenewDocumentCommand \Re {} {\operatorname{Re}}
  \RenewDocumentCommand \Im {} {\operatorname{Im}}
}
\usepackage{mleftright}
\setlength{\delimitershortfall}{-1sp}

%Sprache----------------------------------------------------------
\usepackage{microtype}
\usepackage{xfrac}
\usepackage[autostyle]{csquotes}    % babel
\usepackage[unicode, pdfusetitle]{hyperref}
\usepackage{bookmark}
\usepackage[shortcuts]{extdash}
%Einstellungen hier, z.B. Fonts
\usepackage{booktabs} % Tabellen

%Defininierte funktionen
\DeclareMathOperator{\f}{xyz}

\ExplSyntaxOn % bequeme Syntax für Definition von Befehlen

\NewDocumentCommand \I {} {         %Befehl \I definieren,keine Argumente
  \symup{i}                         %Ergebnis von \I
} 
\NewDocumentCommand \dif {m} % m = mandatory (Pflichtargument für \dif)
{
  \mathinner{\symup{d} #1}
}

\ExplSyntaxOff % Syntax wieder ausschalten. Wichtig!


\begin{document}
    
    Das ist eine Variable $x$ und das ist $\alpha$

    %viele Sache sind in der Matheumgebung wörtlich
    $\forall \alpha \in \Omega \subset \Beta$


    $3 \neq 4$

%Hoch-/runterstellen von Zahlen, sobald die zaheen mehrstellig sind --> x^{...}
    $x^2$ und $x_2$ und $x^{300}$

% einen Text runterstellen ist nicht das gleiche wie etwas runterzustellen
    richtig: $x_\text{min}$
    
    falsch: $x_{min}$

%Codes
    $x \sin (y)$

%Eigene "Funktionen" definieren
    %Möglichkeit 1: In der Mathe Umgebung
    $\operatorname{xyz}_i(a)$
    $\operatorname*{xyz}_i(a)$
    
    %Möglichkeit 2: Open definieren für das gesamte Doc
    $\f^2$ 

    \begin{equation}
        \nabla \cdot \symbf{E} = \frac{\rho}{\varepsilon_0}
        \label{eqn:maxwell1}
    \end{equation}
    Schon Gauß hatte das Durchflutungsgesetz
    \eqref{eqn:maxwell1} aufgestellt.

%Brüche im Text
    Wenn ich einen Bruch im Text schreiben will mache ich es so \sfrac{1}{2} oder so \sfrac{$\symup{\pi}$}{2}
    und nicht so $\frac{1}{2}$

% Links und Email
Wenn man eine Suchmaschine benutzen will bietet sich \href{www.google.de}{Google} an.
Bei Fragen oder Problemen schreib einfach eine Mail an \href{mailto:max@mustermann.de}{max@mustermann.de}.
 
    
\end{document}