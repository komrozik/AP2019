\newpage
\section{Diskussion}
\label{sec:Diskussion}
\subsection{Zeitabhängigkeit der Amplitude}
Die Abhängigkeit der Amplitude von der Zeit hat eine exponentielle abklingende Form und lässt sich damit gut an eine e-Funktion annähern.
Allerdings stimmt die gemessene Zeitabhängigkeit nicht mit der Theoriekurve überein, was durch Probleme bei dem Ablesen der Werte vom Oszilloskop
und durch einen Fehler in der Skalierung der Werte zu erklären ist.
\subsection{Aperiodischer Grenzfall}
Der theoretisch berechnete Wert für den Wiederstand um den aperiodischen Grenzfall zu erreichen beträgt $R_{theo}=1673,32 \Omega$.
Mit der Messung und Betrachtung des Bildes ließ sich der experimentelle Wert für den Wiederstand auf $R_{exp} = 1475\Omega$ bestimmen.
Die gemessene Wert weicht um circa $15\%$ von dem theoretisch berechneten Wert ab,diese Abweichung kann durch die inneren Wiederstände des Aufbaus und durch Störungen am Messfühler erklärt werden.

\subsection{Frequenzabhängigkeit der Kondensatorspannung}
Es ist zu vermuten das sich ein Hochpunkt bei ca. $35kHz$ befindet.
Es hätten noch mehr Werte gemessen werden müssen um sicher das Maximum bestimmen zu können.