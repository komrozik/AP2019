\newpage
\section{Diskussion}
\label{sec:Diskussion}
\subsection{Zeitabhängigkeit der Amplitude}
Die Abhängigkeit der Amplitude von der Zeit hat eine exponentielle abklingende Form und lässt sich damit gut an eine e-Funktion annähern.
Allerdings stimmt die gemessene Zeitabhängigkeit nicht mit der Theoriekurve überein, was durch Probleme bei dem Ablesen der Werte vom Oszilloskop
und durch einen Fehler in der Skalierung der Werte zu erklären ist.\\
Es stellt sich heraus, dass die Dämpfungskoeffizenten (hier: der Widerstand) im Argument der e-Funktion \ref{eqn:efunktion} um einen Faktor
von ca. 2 zu klein ist, sodass die Theoriekurve nicht ganz auf die Messung passt.\\
Grund dafür sind ignorierte Faktoren, wie z.\;B. den kapazivien Einfluss der Spule sowie den
Widerstand der Kabel selbst.



\subsection{Aperiodischer Grenzfall}
Der theoretisch berechnete Wert für den Wiederstand um den aperiodischen Grenzfall zu erreichen beträgt $R_{theo}=1673,32 \Omega$.
Mit der Messung und Betrachtung des Bildes ließ sich der experimentelle Wert für den Wiederstand auf $R_{exp} = 1475\Omega$ bestimmen.
Die gemessene Wert weicht um circa $15\%$ von dem theoretisch berechneten Wert ab. Diese Abweichung kann durch die inneren Wiederstände des Aufbaus, durch Störungen am Messfühler
sowie eine eine ungenaue Abmessung am Oszilloskop erklärt werden.

\subsection{Frequenzabhängigkeit der Kondensatorspannung}
Beim Auftragen der Kondensatorspannung $U_C$ gegen die Frequenz $\omega$ zeigt sich 
ein glockenkurven ähnliches Verhalten.
Es ist zu vermuten, dass sich der Hochpunkt bei ca. $35kHz$ befindet, was die Resonanzfrequenz darstellt,
die Eigenfrequenz des Schwingkreises. Es hätten noch mehr Werte gemessen werden 
müssen, um sicher das Maximum bestimmen zu können.
Aufgetragen wurde dies in einer linearen Verhältnis.
Dabei müssen die Frequenzen mit einem Faktor 5 multipliziert werden, damit sich die
Theoriekurve dem gemessen Umstand anähert. Dies ist auf eine fehlerhafte Einstellung
am Frequenzgenerator zurückzuführen.
\\
Die Frequenzabhängigen Wiederstände nehmen mit tiefen Frequenz zu und fallen für hohen
Frequenzen experimentelle ab.
\\
Der Phasenunterschied im Frequenzbereich von 29 bis 42kHz zeigt nur sehr kleine 
Phasenunterschiede auf. Zu vermuten ist, dass die Phasenverschiebung erst bei tieferen
Frequenz einiger 100Hz überhaupt bemerktbar ist.\\
Da die Spule tiefe Frequenzen einfacher passieren lässt und der Kondensator dagegen
hohe Frequenzen passieren lässt, resultiert daraus eine Verzögerung für tiefe Frequenzen.
Hier müsste eine breites Frequenzspektrum betrachtet werden.  