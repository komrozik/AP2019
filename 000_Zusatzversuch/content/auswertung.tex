\newpage
\section{Auswertung}
\subsection{Wertetabelle}
\begin{table}[H]
    \centering
    \begin{tabular}{c | c c c c c}
        \toprule
        & Feder 1 & Feder 2 & Feder 3 & Feder 4 & Feder 5 \\ & (Basisfeder)\\
        \midrule
        $m\;/\;$g Stk. & 1.3456 & 1.311 & 1.3812 & 1.4574 & 1.2342 \\
        \midrule
        $D_a\;/\;$mm & 3.68 & 3.57 & 3.82 & 3.69 & 3.69 \\
          & 3.67 & 3.57 & 3.81 & 3.68 & 3.69 \\
          & 3.69 & 3.57 & 3.82 & 3.68 & 3.68 \\
          & 3.68 & 3.57 & 3.82 & 3.68 & 3.68 \\
          & 3.69 & 3.57 & 3.82 & 3.76 & 3.68 \\
          & 3.68 &         &         &         &         \\
        \midrule
        $\bar{D_a}\;/\;$mm & 3.68* & 3.57 & 3.818 & 3.69 & 3.684\\
        $D_{a_S}\;/\;$mm& 0.007 & 0 & 0.004 & 0.03 & 0.005\\
        \midrule
        $L0$ 	& 57.04 	& 57.3 		& 56.38 	& 61.15 	& 52.73 	\\
                & 57.17 	& 57.39 	& 56.69 	& 61.22 	& 52.81 	\\
                & 57.07 	& 57.27 	& 56.6 		& 61.31 	& 52.86 	\\
                & 57.16 	& 57.41 	& 56.66 	& 61.29 	& 52.97 	\\
                & 56.89 	& 57.27 	& 56.89 	& 61.2 		& 52.82 	\\
                & 57.04 	&         	&         	&         	&   \\
        \midrule
        $\bar{L0}\;/\;$mm & 57.06 	& 57.33 	& 56.64 	& 61.23 	& 52.84	\\
        $L0{_S}\;/\;$mm & 0.093 	& 0.060 	& 0.16 	& 0.059 	& 0.078	\\
        $n_{wirk}$ & 109.36 & 109.98 & 108.39 & 119.07 & 99.54 \\(berechnet)\\
        \midrule
        $F1\;/\;$N bei $L1=105\;$mm & 4.8 & 5.21 & 4.46 & 4.26 & 5.5\\
                         & 4.86 & 5.28 & 4.48 & 4.25 & 5.5\\
                         & 4.83 & 5.29 & 4.5 & 4.3 & 5.48\\
                         & 4.83 & 5.28 & 4.49 & 4.28 & 5.47\\
                         & 4.85 & 5.29 & 4.52 & 4.23 & 5.49\\
        \midrule
        $F2\;/\;$N bei $L2=142\;$mm & 7.97 & 8.7 & 7.33 & 7.2 & 8.93\\
                         & 8.02 & 8.75 & 7.33 & 7.17 & 8.92\\
                         & 7.96 & 8.73 & 7.35 & 7.22 & 8.93\\
                         & 8.0 & 8.75 & 7.35 & 7.2 & 8.9\\
                         & 8.04 & 8.77 & 7.4 & 7.14 & 8.93\\
        \midrule
        $\bar{F1}\;/\;$N & 4.83 & 5.27 & 4.49 & 4.264 & 5.49\\
        $F1_S\;/\;N$     & 0.021& 0.03 & 0.02 & 0.0241& 0.012\\ 
        $\bar{F2}\;/\;$N & 7.99 & 8.74 & 7.35 & 7.19 & 8.92\\
        $F2_S\;/\;N$     & 0.03 &0.024 & 0.026& 0.028& 0.017\\ 
        \midrule
        $R$ & 0.086 & 0.094 & 0.077 & 0.079 & 0.093\\
        \midrule
        $F_0\;/\;$N & 0.73 & 0.79 & 0.75 & 0.81 & 0.65\\ 
        \bottomrule
    \end{tabular}
    \caption{$m$ Federmasse,
             $D_a$ Federaußendruchmesser, 
             $\bar{D_a}$ Mittelwert des Federaußendruchmesser, 
             $D_{a_S}$ Standartabweichung des Federaußendruchmesser,
             $L_0$ Federlänge,
             $\bar{L}$ Mittelwert der Federlänge,
             $L_{0_S}$ Standartabweichung der Federlänge,
             $n_{wirk}$ wirkende Windungszahl,
             $F1$,$F2$ Federkräfte,
             $\bar{F1}$,$\bar{F2}$ Mittelwert der Federkräfte,
             $R$ Federkonstante,
             $F0$ Innere Vorspannkraft
    }
    \label{tab:Wertetabelle}
\end{table}


\subsection{Variable Federdicke}
Im Folgenden werden die wirkenden Windung $n_{wirk}$ als konstant mit $n_1=109.4$ angenommen, 
da die maximale Abweichung von $\Delta n_{max}=0.97$ als hinreichend klein angenommen wird.


\begin{figure}[H]
    \center
    \includegraphics[width=0.8\textwidth]{plots/D_kraftweg_dia.pdf}
    \caption{Feder 1,2,3 mit unterschiedlichen Federaußendurchmessern $D_a$ bei konstanter Basiswindungszahl $n$.
    Aufgetragen in einem Kraft-Weg-Diagramm. Gemessen wurden dabei
    die für L1 und L2 resultierenden Federkräfte F1 und F2.}
    \label{tab:LF_D}
\end{figure}
\subsection{Lineare Ausgleichsgerade}
\label{sec:fit}
Für die jeweiligen Ausgleichsgeraden aus \ref{tab:LF_D} wird ein Polyfit \cite{numpy_polyfit}
ersten Grades durchgeführt und daraus die Steigung, also die Federkonstante $R$ so wie eine Verschiebung
in der vertikalen $v$ ermittelt.\\
Aus \ref{eqn:federrate} mit
\begin{align*}
  F&=R \cdot L + v ,\\
  F&=R \cdot s + F0
\end{align*}
folgen somit die in \ref{tab:Wertetabelle} aufgeführten Federkonstanten.
\begin{align*}
  R1= 0.086\;\si{\N\per\mm}, &&  v1= -4.14\;\si{\N},\\
  R2= 0.094\;\si{\N\per\mm}, &&  v2= -4.58\;\si{\N},\\
  R3= 0.077\;\si{\N\per\mm}, &&  v3= -3.63\;\si{\N}.\\
\end{align*}

\subsection{Innere Vorspannkraft ermitteln}
\label{sec:vorspannkraft}
Um nun aus den Verschiebungswerten $v$ auf die innere Vorspannkraft $F_0$ verschiebe
man die Gerade so, dass nun der Federweg $s$ betrachtet wird.
Die Kraft $F_0$, die zuvor bei der Federlänge $L_0$ lag, liegt nun bei $s=0$
und beschreibt die zugehörige Vorspannkraft $F_0$.
\begin{align*}
  F&=R \cdot L + v=R\cdot L0+v\\
  R&=R \cdot s + (v+L0 \cdot R) \text{ mit }F0:= (v+L0 \cdot R) \\
\end{align*}
\begin{figure}[H]
  \center
  \includegraphics[width=0.8\textwidth]{plots/f0_123_dia.pdf}
  \caption
  {
    Federkraft $F$ betrachtet für Federweg $s$, y-Achsenabschnittet bildet dabei die innere Vorspannkraft $F_0$.
    So folgt aus einer Geradengleichung mit $y_0=R\cdot(L-L_0)+b$ dass $y_{\Delta L}=y_0+R \cdot L_0=R \cdot L+b$
    wobei für $L=0$ folglich gilt $y_{\Delta L}(L=0)=b=:F_0$.
  }
\end{figure}
Somit ergeben sich für die jeweilligen inneren Vorspannkräfte $F0$
\begin{align*}
  F0_1=  0.73 \;\si{\N},\\
  F0_2=  0.79 \;\si{\N},\\
  F0_3=  0.75 \;\si{\N}.\\
\end{align*}



\begin{figure}[H]
  \center
  \includegraphics[width=0.8\textwidth]{plots/dicke_konstante_dia.pdf}
  \caption{Die Federdicke $D_a$ gegen die Federkonstante $R$ aufgetragen.}
  \label{fig:R_D_dia}
\end{figure}

\subsubsection{Ausgleichsgerade}

Es folgt aus \ref{eqn:federrate}
\begin{align*}
  R=\frac{G\;d^4}{8\;n}\cdot \frac{1}{D^3}, \\\\  
  \text{mit }k_D =\frac{G\;d^4}{8\;n},
\end{align*}
Es folgt die Funktion
\begin{equation*}
  R(D)=k_D \cdot \frac{1}{D^3},
\end{equation*}
mit dem Parameter
\begin{equation*}
  k_D=(2.94 \pm 0.03) \;\si{\N\mm\squared}
\end{equation*}


\subsection{Variable Federwindungszahl}
\begin{figure}[H]
    \center
    \includegraphics[width=0.8\textwidth]{plots/n_kraftweg_dia.pdf}
    \caption{Feder 1,4,5 mit unterschiedlichen Windungszahlen $n$ bei konstanter Basis Federdicke $D_a$.
    Aufgetragen in einem Kraft-Weg-Diagramm. Gemessen wurden dabei die für L1
    und L2 resultierenden Federkräfte F1 und F2}
\end{figure}
Analog wie in Abschnitt \ref{sec:fit} folgt für die Federkonstante $R$
und den Verschiebungswert $v$ aus dem mit der Methodik aus \ref{sec:vorspannkraft}
die innere Vorspannkraft $F_0$ ermittelt wird
\begin{align*}
  R1= 0.086\;\si{\N\per\mm}, &&  v1= -4.14\;\si{\N}, && F1_0=0.73\;\si{\N}\\
  R4= 0.079\;\si{\N\per\mm}, &&  v4= -4.03\;\si{\N}, && F4_0=0.81\;\si{\N}\\
  R5= 0.092\;\si{\N\per\mm}, &&  v5= -4.26\;\si{\N}. && F5_0=0.65\;\si{\N}\\
\end{align*}

\begin{figure}[H]
  \center
  \includegraphics[width=0.8\textwidth]{plots/n_konstante_dia.pdf}
  \caption{Einfluss der Windungszahl $n_{wirk}$ auf die Federkonstante $R$}
  \label{fig:R_n_dia}
\end{figure}

\subsubsection{Ausgleichsgerade}

Es folgt aus \ref{eqn:federrate}
\begin{align*}
  R=\frac{G\;d^4}{8\;D^3}\cdot \frac{1}{n_{wirk}} \\\\  
  \text{mit } k_n=\frac{G\;d^4}{8\;D^3}
\end{align*}
Es folgt die Funktion
\begin{equation*}
  R(n)=k_n \cdot \frac{1}{n_{wirk}}
\end{equation*}
mit dem Parameter
\begin{equation*}
  k_n=(9.32 \pm 0.05) \;\si{\N\per\mm}
\end{equation*}


\subsection{Betrachtung der Masse}
\begin{figure}[H]
  \center
  \includegraphics[width=0.8\textwidth]{plots/masse_konstante_dia.pdf}
  \caption{Massenresulate aus der Variation der Federdicke $D_a$
          und der Windungszahl $n$ der Federkonstanten $R$ gegenübergestellt. }
\end{figure}
\begin{figure}[H]
  \center
  \includegraphics[width=0.8\textwidth]{plots/m_D_dia.pdf}
  \caption{}
\end{figure}
\begin{figure}[H]
  \center
  \includegraphics[width=0.8\textwidth]{plots/m_n_dia.pdf}
  \caption{}
\end{figure}

\label{sec:Auswertung}
