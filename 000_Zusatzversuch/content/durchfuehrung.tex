\newpage
\section{Durchführung}
Folgendes soll untersucht werden:
\begin{enumerate}
    \item Messung: Einfluss der Windungszahl $n$ auf die Federkraft $F$. Dies soll dem zusätzlichen
    Materialverbrauch gegenübergestellt werden.
    \item Messung: Einfluss der Federdicke $d$ auf die Federkraft $F$. Dies soll dem zusätzlichen
    Materialverbrauch gegenübergestellt werden.
    \item Fazit: Kompinierung der Parameter Windungszahl $n$ und Federdicke $d$ für ein minimum
    an Materialverbrauch bei vorgegebener Federkrafttoleranzbereich.
\end{enumerate}
Als Basis verwende man eine zu der Zeichnung und den Größen passende Zugfeder. Diese
wird zunächst gewogen (mehrere Federn und dann mitteln) und ihre Federkraft $F_{Basis}$
geprüft.\\
Nun werden jeweils X Federn mit varriierender Windungszahl $\Delta n$ produziert und deren
Gewicht sowie Federkraft $F_{\Delta n}$ vermessen.
Danach wird die Maschine auf die Basislage zurückgestellt. NUn wird der Paramter 
der Federdicke um $\Delta d$ varrieriert und resultierende Federkraft $F_{\Delta d}$
vermessen.  
\label{sec:Durchfuehrung}