\newpage
\section{Durchführung}
\subsection{Fouriersynthese 1}
    Um den Plot für $f_1(x) = |\sin(x)|$ zu erstellen muss man zunächst entscheiden ob die Funktion gerade($f(x)=f(-x)$) oder ungerade ist.
    Da $f_1(x)$ eine gerade Funktion kommen keine Sinus-Terme in der Fourierreihe vor.
    Somit ist $B_k = 0$ und es müssen die jeweiligen $A_k$ berechnet werden.
    Außerdem muss zum lösen des Integrals eine passende Periode gewählt werden.
    \begin{align}
        A_k &= \frac{2}{\pi}\int_{-\pi/2}^{+\pi/2} |\sin(x)|\cos(\omega_k x)dx & mit\; A_0 &= \frac{1}{\pi}\int_{-\pi/2}^{+\pi/2} |\sin(x)|dx  \label{eqn:Aksin} 
    \end{align}
    Um das Integral in \eqref{eqn:Aksin} zu lösen muss $|\sin(x)|$ für die Intevalle von $-\frac{\pi}{2}$ bis 0 und für 0 bis $\frac{\pi}{2}$ definiert werden.
    \begin{gather}
        A_k = \frac{2}{\pi}\int_{-\pi/2}^{0} -\sin(x)\cos(\omega_k x)dx + \frac{2}{\pi}\int_{0}^{+\pi/2} \sin(x)\cos(\omega_k x)dx \\
        mit \; A_0 = \frac{1}{\pi}\int_{-\pi/2}^{0} -\sin(x)dx +\frac{1}{\pi}\int_{0}^{+\pi/2} \sin(x)dx
    \end{gather}

\subsection{Fouriersynthese 2}
    Die Funktion $f_2(x) = x$ ist eine ungerade Funktion und ist auf dem Intervall von $-\pi$ bis $\pi$ definiert.
    Daher ist die Periodendauer $T = 2\pi$ und die Kosinus-Terme fallen weg und es gilt $A_k = 0$.\\
    Für \eqref{eqn:Bk} gilt nun:
    \begin{equation}
        B_k = \frac{1}{\pi}\int_{-\pi}^{+\pi} x\cos(\omega_k x)dx
    \end{equation}
\label{sec:Durchführung}
