\section{Theorie}
\label{sec:Theorie}
Die Fourier Synthese ist eine Methode um eine beliebige Funktion mit einer Summe von Kosinus und Sinus Funktionen zu approximieren.
\begin{align}
    f(t) &= \sum_{k=0}^{\infty} (A_k\cos(\omega_k t)+B_k\sin(\omega_k t)) &   mit \; \omega_k &= \frac{2\pi k}{T}
\end{align}
wobei die Koeffizienten $A_k$ und $B_k$ festgelegt sind durch
\begin{align}
    A_k &= \frac{2}{T} \int_{-T/2}^{+T/2} f(t)\cos(\omega_k t) dt    &  mit \; A_0 &= \frac{1}{T} \int f(t) dt \\
    B_k &= \frac{2}{T} \int_{-T/2}^{+T/2} f(t)\sin(\omega_k t) dt    &  mit \; B_0 &= 0
\end{align}
\\
Dabei beschreibt $T$ die Periodendauer, $\omega_k$ die Frequenz und $A_k$ und $B_k$ die zugehörigen Koeffizienten. \\
Im Folgenden soll mit dem Programm Fouriersynthese %Bib setzen
die Funktionen $f(x)=x$ und $f(x)=|sin(x)|$ nachgestellt und die jeweilligen Koeffizienten $B_k$ und $A_k$ ermittelt werden.
