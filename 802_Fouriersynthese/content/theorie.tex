\section{Theorie}
\label{sec:Theorie}
Die Fourier Synthese ist eine Methode um eine beliebige Funktion mit einer Summe von Kosinus und Sinus Funktionen zu approximieren.
\begin{gather}
    f(t) = \sum_{k=0}^{\infty} (A_k\cos(\omega_k t)+B_k\sin(\omega_k t))\\
    \omega_k = \frac{2\pi k}{T}
\end{gather}
wobei die Koeffizienten $A_k$ und $B_k$ festgelegt sind durch
\begin{gather}
    A_k=\\
    B_k=\\
\end{gather}
\\
Dabei beschreibt $T$ die Periodendauer, $\omega_k$ die Frequenz und $A_k$ und $B_k$ die zugehörigen Koeffizienten. 

Im Folgenden soll mit dem Programm Fouriersynthese %Bib setzen
die Funktionen $f(x)=x$ und $f(x)=|sin(x)|$ nachgestellt und die jeweilligen Koeffizienten $B_k$ und $A_k$ ermittelt werden.
\cite{sample}
