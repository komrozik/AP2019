\section{Theorie}
\label{sec:Theorie}
Die Fourier Synthese ist eine Methode um eine beliebige Funktion mit einer Summe von Kosinus und Sinus Funktionen zu approximieren.
Die Theorie besagt dass man jede Funktionen mit einer unendlichen Anzahl von Trigonometrischen Funktionen komplett nachbauen kann.
\begin{gather}
    f(t) = \sum_{k=0}^{\infinity} (A_k\cos(\omega_k t)+B_k\sin(\omega_k t))\\
    \omega_k = \frac{2\pi k}{T}\\
    
\end{gather}
\cite{sample}
