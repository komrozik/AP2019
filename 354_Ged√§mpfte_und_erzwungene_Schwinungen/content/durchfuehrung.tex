\newpage
\section{Durchführung}
\label{sec:Durchfuehrung}
\subsection{Aufbau}
\subsection{Messungen}
Der Versuch ist in 4 Teilmessungen aufgeteilt.
Im ersten Teil des Versuchs wird die Zeitabhängigkeit der Amplitude untersucht.
Um diese Zeitabhängigkeit festzustellen wird eine Rechtecksschwingung über einen Wiederstand $R_1$ ,eine Spule $L$ und einen Kondensator $C$ gesendet (\ref{fig:Aufbau1}).
Mit dem Ozilloskop wird die Spannung über dem Kondensator abgenommen und das Bild des Oszilloskops wird abfotografiert (\ref{fig:Bild1}) und analysiert.
\begin{figure}
\caption{Versuchsaufbau - Zeitabhängigkeit der Amplitude}
\label{fig:Aufbau1}
\end{figure}

Im zweiten Teil des Versuchs soll der Grenzwiederstand $R_{ap}$ bestimmt werden bei dem der aperiodische Grenzfall eintritt.


Im dritten Teil des Versuchs wird die Frequenzabhängigkeit der Kondensatorspannung an einem Serienresonanzkreis gemessen.
Der Serienresonanzkreis ist eine Schaltung bei der  der Generator ein Sinussignal durch den größeren der beiden Wiederstande $R_2$, durch die Spule $L$ und durch einen Kondensator $C$ leitet.
Wieder wird das Signal parallel zum Kondensator abgenommen um die Spannung zu messen.
Für die Messung wird zuerst die Frequenz bestimmt bei der die höchste Spannung gemessen wird, bei Schaltkasten 3 lag diese Frequenz ungefähr bei $f = 35 \text{kHz}$.
Nun werden jeweils 7 ganzzahlige Werte unter und über dieser Frequenz eingestellt und die entsprechende Spannung abgelesen.

Im vierten Teil des Versuchs wird die Abhängigkeit der Phase (zwischen Erreger- und Kondensatorspannung) von der Frequenz bestimmt.
Es wird die Kondensatorspannung wie in Teil drei des Versuchs gemessen und zusatzlich wird noch das Erregersignal abgenommen und im Oszilloskop dargestellt.
Anhand der 2 Kurven im Oszilloskop kann der Phasenunterschied durch den Abstand der Minima der Kurven abgelesen werden.
