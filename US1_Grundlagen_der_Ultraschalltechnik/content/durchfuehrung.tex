\section{Durchführung}
\label{sec:Durchfuehrung}
\subsection{Vorbereitung}
Vermessen werden drei Acyrilzylinder mit verschieden Längen. Genutzt wird dabei das Impuls-
Echo-Verfahren und das Durchschall-Verfahren.\\
Zunächst werden die Längen der Zylinder mithilfe der Schieblehre vermessen.
\subsection{Verwendung Impuls-Echo-Verfahren}
Verwendet wird eine 2Mhz-Sonde die als Sender und Empfänger fungiert.
Die Acyrilzylinder werden auf ein PApiertuch gestellt, um Beschädigungen zu vermeiden.
Anschließend wird die Sonde mit bidestilliertem Wasser an den Zylinder gekoppelt.\\

Über die Abschwächung der Intensität des reflektierten Puls, kann auf den Absorbtionskoeffizienten bzw.
die Dämpfung zurückgeschlossen werden.\\
Anschließend kann über den A-Scan die Laufzeit des Echos ermittelt werden. 
Daraus folgert sich die Schallgeschwindigkeit im Acryil.
Über eine Ausgleichsgerade kann der systematische Fehler der aufgrund der Anpassungsschicht der Sonde verursacht wird,
zurückgeschlossen werden.
\subsection{Verwendung Durchschallungs-Verfahren}
Die Acrylzylinder werden nun horizontal in eine schwarze Halterungen gespannt und 
die 2Mhz-Sonde mit Koppelgel an die Stirnseite gekoppelt.\\
Über einen A-Scan kann nun die LAufzeit bestimmt werden und daraus ddie Schallgeschwindigkeit berechnet werden.
\subsection{Biometrische Untersuchung eines Augenmodells}
Die 2Mhz-Sonde iwrd mit Koppelgel an die Hornhaut des Augenmodells gekoppelt.
Aus dem A-Scan kann nun über das Echo der Grenzflächen von Iris und Retina die LAufzeit und 
somit die Abmessungen des Auges berechnet werden. 

