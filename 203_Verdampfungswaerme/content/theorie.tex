\newpage
\section{Theorie}
\label{sec:theorie}
Ziel: Bestimmung der Dampfdruckkurve im bereich ca. 30-1000mbar.
Zusätzlich soll die zugehörigen Verdampfunswärme $L$ und dessen temperaturabhängigkeit bestimmt werden.\\

\subsection{Mikroskopische Vorgänge}
In einer Flüssigkeit bewegen sich die Teilchen mit der Maxwell'schen Geschwindigkeitsverteilung.
Einige Moleküle haben genug kinetische Energie um die Flüssigkeit zu verlassen.
Dabei verrichten sie Arbeit, dessen Energie auf außen durch z.B. Erhitzen hinzugefügt werden muss.

\begin{figure}
    \includegraphics[]{%Kessel mit Teilchen}
\end{figure}

Betrachtet wird der Phasenübergang von Wasser.
Dabei haben generell die Phasen (fest,flüssig,gasförmig) zwei Freiheitsgerade - die Temperatur $T$ und den Druck $P$.
Dabei gibt es Bereiche in denen zwei Phasen nebeneinander vorliegen. Hierbei sind die Termperaturen $T$ und der Druck $P$
nicht frei wähltbar. Es liegt nur noch ein Freiheitsgrad.\\
Diese Bereiche werden durch die Dampfdruck-Kurve beschrieben, die die Phasen von einander abgrenzt.
Auf dieser Kurve gibt es zwei charakteristische Punkt:
\beginn{figure}
\includegraphics[width=0.7\textwidth]{%Plot aus Anleitung}
\end{figure}

\begin{enumerate}
    \item Trippel-Punkt (T.P.): Beschreibt den Zustand in denen alle drei Aggregardszustände neben einander vorliegen.
    \item Kritischer Punkt (K.P.): Ist das Ende der Dampfdruckkurve, und beschreibt den Zustand in dem es keinen Unterschied
        zwischen dem Aggregatszustand flüssig und gasförmig nicht mehr unterchieden werden kann.
\end{enumerate}

\subsection{Verdampfungswärme $L$}
Die Dampfdruckkurve wird bestimmt durch die Verdampfungswärme $L$.
Diese Größe ist eine charakteristische Größe der Substanz und im allgemeinen temperaturabhängig.
Allerdings verhält sie sich in Nähe des Trippel Punktes in guter Näherung konstant (temperaturunabhängig).
