\section{Diskussion}
\label{sec:Diskussion}
Für die Messergebnisse im Niedrigdruckbereich wurden die ersten beiden Werte aus der Berechnung entfernt, da nach ihnen ein großer Temperatursprung stattfindet, der aus problemen mit der Messapparatur stammt.
Wenn der somit berechnete Wert aus der Messung im niedrigen Druckbereich mit dem Literaturwert verglichen wird liegt er mit dem Fehlerbereich genau auf dem Literaturwert.

Bei der Messung für den Verdampfungskoeffizient in Temperaturabhängigkeit muss aus physikalischen Gründen der Fall in dem die Wurzel addiert wird betrachtet werden.
Dieser Fall wird als Kurve in Abb. 7 gegen die Temperatur aufgetragen und passt in die pysikalische Vorraussetzung, dass der Verdampfungskoeffizient zum kritischen Punkt hin kleiner werden muss und gegen Null fällt.