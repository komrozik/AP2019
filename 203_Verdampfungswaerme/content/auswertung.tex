\newpage
\section{Auswertung}
\label{sec:Auswertung}
\subsection{Niedrigdruckbereich}
\begin{figure}
    \centering
    \includegraphics[width=0.7\textwidth]{build/plot_a.pdf}
    \caption{Ausgleichskurve für den Bereich unter 1 Bar}
    \label{fig:Niedrigdruck}
\end{figure}
Damit werden aus der Ausgleichsgeraden die Parameter bestimmt:
\begin{align}
    a &= -141\pm 5 \; \text{K}\\
    b &= 0.99\pm 0.09
\end{align}
Nachdem Gleichung (\ref{eqn:logarithm_p}) nach $L$ umgestellt wird folgt für $a$:
\begin{align}
    ln(p &= -\frac{L}{R}\frac{1}{T}+\text{const} // L &= -a\cdot R
\end{align}
Mit dem Fehler nach (???) folgt:
\begin{align}
    L = (1.18 \pm 0.04)\cdot 10^3 \; \frac{\text{J}}{\text{mol}}
\end{align}
\subsection{Hochdruckbereich}
Mit der Gleichung (\ref{eqn:L}) folgt für L:
\begin{align}
    L = T(V_D - V_F) \frac{dp}{dT}
\end{align}
In der Formel wird $V_F$ vernachlässigt und $V_D$ wird mit folgender Formel berechnet:
\begin{align}
    \biggl( p+\frac{a}{V_D^2} \biggr) V &= RT\;\; \text{mit} \; a = 0,9 \; \frac{\text{Joule}\;{\text{m}}^3}{{\text{Mol}^2}}\\
    V &= \frac{RT}{2p} \pm \sqrt{{\frac{RT}{2p}}^2-\frac{a}{p}}
\end{align}