\section{Auswertung}
\label{sec:Auswertung}
\subsection{Niedrigdruckbereich}
\begin{figure}
    \centering
    \includegraphics[width=0.7\textwidth]{build/plot_a.pdf}
    \caption{Ausgleichskurve für den Bereich unter 1 Bar}
    \label{fig:Niedrigdruck}
\end{figure}
Damit werden aus der Ausgleichsgeraden die Parameter bestimmt:
\begin{align*}
    f(x) &= a\cdot x + b \\
    a &= -5148 \pm 15 \; \text{K}\\
    b &= 11.52 \pm 0.05
\end{align*}
Nachdem (\ref{eqn:logarithm_p}) nach $L$ umgestellt wird folgt für $a$:
\begin{align*}
    \ln(p) &= -\frac{L}{R}\frac{1}{T}+\text{const} \\ L &= -a\cdot R
\end{align*}
Daraus folgt:
\begin{align*}
    L = (42.80\pm 0,12)\cdot 10^3 \; \frac{\text{J}}{\text{mol}}
\end{align*}
Aus dieser Verdampfungswärme $L$ soll nun die äußere Verdampfungswärme $L_a$ bestimmt werden.
Dazu muss die Volumenarbeit $W = pV$ mit der idealen Gasgleichung gleichsetzt werden.
Für 100°C ergibt sich:
\begin{align*}
    L_a = pV = RT = 3101,3 \; \frac{\text{J}}{\text{mol}}
\end{align*}
Die dementsprechende innere Energie wird nun über die Differenz von $L$ und $L_a$ berechnet
\begin{align*}
    L_i = L - L_a = (39,70\pm0,12)\cdot 10^3 \; \frac{\text{J}}{\text{mol}}
\end{align*}
Um die genauere innere Energie jedes einzelnen Moleküls zu erhalten muss die gesamte innere Energie noch durch die Avogardo Konstante geteilt werden.
\begin{align*}
    L_i = 0.4115\pm 0.0012 eV
\end{align*}
\subsection{Hochdruckbereich}
Aus (\ref{eqn:L}) folgt für L:
\begin{align*}
    L = T(V_D - V_F) \frac{dp}{dT}
\end{align*}
In der Formel wird $V_F$ vernachlässigt und $V_D$ wird mit folgender Formel berechnet:
\begin{align*}
    \biggl( p+\frac{a}{V_D^2} \biggr) V &= RT\;\; \text{mit} \; a = 0,9 \; \frac{\text{Joule}\;{\text{m}}^3}{{\text{Mol}^2}}\\
    V &= \frac{RT}{2p} \pm \sqrt{{\frac{RT}{2p}}^2-\frac{a}{p}}
\end{align*}
Für den Bereich über 1 Bar ist L temperaturabhängig, daher wird hier p nicht logarithmisch über T geplottet.
Als Fit wird hier ein Polynom 3. Ordnung verwendet.
\begin{figure}
    \centering
    \includegraphics[width=0.7\textwidth]{build/plot_b.pdf}
    \caption{Ausgleichskurve für den Bereich über 1 Bar}
    \label{fig:Hochdruck}
\end{figure}
\\Der Fit in Form von $p(T) = aT^3+bT^2+cT+d$ hat folgende Parameter:
\begin{align*}
    a &= (4,3\pm 1,4)\cdot 10^-6 \; \frac{\text{PA}}{\text{K}^3}\\
    b &= -0,0040\pm 0,0019 \; \frac{\text{PA}}{\text{K}^2}\\
    c &= 1,2\pm 0,8 \; \frac{\text{PA}}{\text{K}}\\
    d &= -1,1\pm 1,1  \; \text{Pa}
\end{align*}
Mit Formel (11) in Formel (9) eingesetzt und den passenden Messwerten ergibt sich für L in Abhängigkeit von der Temperatur
\begin{align*}
    L(T) = \frac{(3aT^3+2bT^2+cT)}{aT^3+bT^2+cT+d}\cdot \biggl( \frac{RT}{2} \pm \sqrt{\frac{RT}{2}^2 - 0,9 \cdot (aT^3+bT^2+cT+d)} \biggr)
\end{align*}
\begin{figure}
    \centering
    \includegraphics[width=0.7\textwidth]{build/plot_c.pdf}
    \caption{Verdampfungswärme nach Temperatur aufgetragen}
    \label{fig:Hochdruck}
\end{figure}
