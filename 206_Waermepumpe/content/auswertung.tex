\newpage
\section{Auswertung}
\label{sec:Auswertung}

\subsection{Temperaturverläufe}
    Die Temperaturverläufe in Abbildung \ref{fig:plot_temp} dargestellt
    \begin{figure}
        \centering
        \includegraphics[width=\textwidth]{build/plot_temp.pdf}
        \caption{Temperaturverläufe}
        \label{fig:plot_temp}
    \end{figure}

\subsection{Nicht-lineare Ausgleichsrechnung}
    Mit der folgenden Näherung, werden nun die in \ref{fig:plot_temp}
    dargestellten Ausgleichsgeraden bestimmt\cite{curvefit}:
    \begin{equation}
        T(t)=At^2+Bt+C
        \label{eqn:ausgleichsgerade}
    \end{equation}
    \begin{table}
        \centering
        \begin{tabular}{c || c | c}
            \toprule
            & $T_1\;/\;K$ & $T_2\;/\;K$ \\
            \midrule
            A\;/\;$K/s^2$& $(-3,8\pm0,9)\;10^{-6}$ & $(1,01\pm0,16)\;10^{-5}$ \\
            B\;/\;$K/s$& 0,0304\pm0,0012 & -0,0339\pm0,0019 \\
            C\;/\;$K$& 293,88\pm0,30 & 298,5\pm0,5 \\
            \bottomrule
        \end{tabular}
    \end{table}
\subsection{Differentialquotienten}
    Exemplarisch werden für vier Messwerte der Differentialquotienten $dT1/dt$ und
    $dT2/dt$ berechnet.
    Für die Näherung von \eqref{eqn:ausgleichsgerade} folgt somit:
    \begin{equation}
        \frac{dT}{dt}=2At+Bt
    \end{equation}
    
    \begin{table}
        \centering
        \begin{tabular}{c c c}
            \toprule
            Zeit $T\;/\;s$ & $T_1$ & d$T_1$/d$t$ \\
            \midrule
            240,0 & 300,35 & 0,0285\pm0,0012\\
            480,0 & 307,55 & 0,0267\pm0,0015 \\
            840,0 & 317,045 & 0,0239\pm0,0020  \\
            1080,0 & 322,045 & 0,0221\pm0,0023\\
            \midrule
            Mittelwert &&  0.0253\pm0.0017  \\
            \bottomrule
        \end{tabular}
        \caption{Differentialquotienten für $T_1$}
        \label{fig:tab_T1t}
    \end{table}

    \begin{table}
        \centering
        \begin{tabular}{c c c c}
            \toprule
            Zeit $T\;/\;s$ & $T_2$ & d$T_2$/d$t$ \\
            \midrule
            240,0 & 291,75 & -0,029 \\
            480,0 & 284,85 & -0,024 \\
            840,0 & 276,55 & -0,017 \\
            1080,0 & 273,95 & -0,012 \\
            \midrule
            Mittelwert &&  -0,021 \\
            \bottomrule
        \end{tabular}
        \caption{Differentialquotienten für $T_2$}
        \label{fig:tab_T2t}
    \end{table}
    \newpage
    \subsection{Bestimmung der Güteziffer}
    Wie in Gleichung (\ref{eqn:nu_ideal}) dargestellt, lässt sich aus dem zuvor berechneten
    Differentialquotienten die Güteziffer $\nu$ bestimmen.\\
    Für die Wärmekapazität von Wasser $m_1\;c_w$ und von Kupfer $m_k\;c_k$ gilt gilt \cite{wasser}:
    \begin{align*}
        m_1\;c_w &= 4190\frac{J}{kg\;K} 	\Rightarrow 12570\frac{J}{K}\\
        m_k\;c_k &= 750\frac{J}{K}
    \end{align*}

    \begin{table}
        \centering
        \begin{tabular}{c c c}
        \toprule
        Zeit $t\;/\;s$ & Güteziffer $v_{ideal}$ & Güteziffer $v_{real}$  \\
        \midrule
        120,0 & 96,02 & 1,97\pm0,08 \\
        480,0 & 24,77 & 1,87\pm0,09 \\
        840,0 & 7,83 & 1,60\pm0,13 \\
        1140,0 & 6,56 & 1,45\pm0,16 \\
        \end{tabular}
    \end{table}
    
%%%%%%%%%%%%%%%%%%%%%%%%%%%%%%%%%%%%%%%%%%%%%%%%%%%%%%%%%

\subsection{Massendurchsatz}
Zuerst wird die Verdampfungswärme mithilfe einer linearen Regression wie in Versuch 203 bestimmt.
Die Ausgleichskurve wird über den folgenden Zusammenhang gebildet.
\begin{align*}
    \ln(\frac{p}{p_0})&= -\frac{L}{R}\cdot\frac{1}{T}\\
    y &= -m \cdot x +b
\end{align*}
Über die Parameter
\begin{align*}
  \symup{m} &= 2237\pm76 \\
  \symup{b} &= 8,06\pm0,25 \:
\end{align*}
lässt sich dann L bestimmen.
\begin{align*}
    L &= m\cdot R\\
    L &= 18604,98 \pm 0,25
\end{align*}
\begin{figure}[H]
  \includegraphics{build/plot_L.pdf}
  \caption{Dampfdruckkurve für $\symup{P}$ und $\symup{T}$ im warmen Reservoir.}
  \label{fig:Dampfp}
\end{figure}
Die Unsicherheit ergibt sich mit der Gauß'schen Fehlerfortpflanzung.
Mit dem so erhaltenen Wert für $L$ wird nun der Massendurchsatz bestimmt.
\begin{equation*}
    \frac{\symup{d}m}{\symup{d}t} = \frac{\nu_{real}\cdot N}{L}
\end{equation*}
Der Massendurchsatz $\symup{d}m / \symup{d}t$ an der jeweiligen Messstelle ist
in Tabelle \ref{tab:massendurch} aufgeführt.
\begin{table}[H]
        \centering
        \begin{tabular}{c c}
        \toprule
        Zeit $t\;/\;s$ & Massendurchsatz in kg/s \\
        \midrule
        120,0 & 0,0211\pm0,0008\\
        480,0 & 0,0201\pm0,0009\\
        840,0 & 0,0171\pm0,0014 \\
        1140,0 & 0,0155\pm0,0017 \\
        \end{tabular}
        \caption{Massendurchsatz}
        \label{tab:massendurch}
    \end{table}

\subsection{Mechanische Kompressionsleistung}
Die mechanische Kompressionsleistung bestimmt sich über \ref{eqn:kompress2}
\begin{table}[H]
        \centering
        \begin{tabular}{c c}
        \toprule
        Zeit $t\;/\;s$ & Kompressorleistung in ?? \\
        \midrule
        120.0 & 0.00143\pm0.00006\\
        480.0 & 0.00139\pm0.00006\\
        840.0 & 0.00121\pm0.00010 \\
        1140.0 & 0.00109\pm0.00012 \\
        \end{tabular}
        \caption{Kompressorleistung}
        \label{tab:kompress}
    \end{table}